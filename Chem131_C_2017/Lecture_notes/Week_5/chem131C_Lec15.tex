\documentclass{article}
\usepackage[utf8]{inputenc}

\title{Chem-131C-Lec15}

\author{swflynn }
\date{May 2017}

\usepackage{natbib}
\usepackage{graphicx}
\usepackage{braket}
\usepackage{amsmath}
\usepackage[margin=0.7in]{geometry}
\usepackage{subfigure}
\usepackage{url}
\usepackage{float}

\begin{document}

\maketitle

\section*{Lecture 15; 5/5/17}
This lecture (and all subsequent lectures) will not be on the midterm, but will appear on the final.

\subsection*{Free Energy Thermodynamic Potentials}
As we have mentioned previously, other Thermodynamic Potentials besides the Potential Energy and Enthalpy exist. 
The \textbf{Helmholtz} Free Energy A(T,V), is given the symbol A from German for 'arbeit' meaning work. 
The \textbf{Gibbs} Free Energy G(T,P), is no surprise named after the man. 

The reason we have different thermodynamic potentials is convenience. 
Physical processes are usually done under different conditions (constant Pressure or Temperature, or Volume, etc). 
It turns out that the variables we need to consider for these different conditions can be easily manipulated using  different Thermodynamic Potentials.

Recall from previous courses that we can write the change in Gibbs Energy (see Lecture 7 for Thermodynamic Potential Definitions) as
\begin{equation}
    \Delta G = \Delta H - T\Delta S
\end{equation}
As you can guess, this is already assuming a constant Temperature process. 
\begin{equation}
    \begin{split}
        G &\equiv A + PV \\
        &= U - TS + PV = U + PV - TS \\
        &= H - TS \\
        \Delta G &= \Delta H - \Delta(TS) \xrightarrow{\text{$\Delta$T=0}} \\
        \Delta G &= \Delta H -T\Delta S
    \end{split}
\end{equation}
If we further assume that $\Delta$H = 0 we see that 
\begin{equation}
    \Delta G = - T\Delta S
\end{equation}
From here we know that a spontaneous process must have an increase in entropy from the second law, therefore the Gibbs Free Energy must be negative during a spontaneous process (Given $\Delta$T and $\Delta$H  = 0). 
This limiting case suggests the sign of $\Delta$G must be negative for any process to be spontaneous (due to the exact differentials of state functions), therefore a process that decreases entropy can be spontaneous, only if the Enthalpy compensates. 

We now recall some terms from general chemistry \textbf{Exothermic} and \textbf{Endothermic}.
An exothermic process is one where heat is released (-$\Delta$H), and the endothermic is the opposite. 
We can use the enthalpy to keep track of the environments entropy (recall the entropy of the environment only depends on heat transfer and the temperature). 
So if we have a -$\Delta$G we have already accounted for the second law and subsequently have a spontaneous process!

\subsubsection*{}
Note this Lecture is extremely short because the beginning discussed the Carnot Engine. 
I compressed these notes into the previous lecture for midterm studying convenience. 

\end{document}