\documentclass{article}
\usepackage[utf8]{inputenc}

\title{Chem132A: Lecture 10}
\author{Shane Flynn (swflynn@uci.edu) }
\date{10/20/17}


\usepackage{graphicx}
\usepackage{amsmath}
\usepackage{braket}
\usepackage[margin=0.7in]{geometry}
\usepackage{hyperref}


\newcommand{\be}{\begin{equation}}
\newcommand{\ee}{\end{equation}}
\newcommand{\benum}{\begin{enumerate}}
\newcommand{\eenum}{\end{enumerate}}
\newcommand{\pd}{\partial}

\begin{document}

\maketitle

\section*{Exam 1 Logistics}
Exam 1 will be on October 25 (Wednesday)!
The exam will contain 4-6 questions, you will have 45 minutes to complete the exam.
Students are allowed a calculator, and a single 8.5 $\times$ 11 in. sheet of paper to write whatever you like (on both sides). 
There will be a seating chart, and bring your student ID to the exam.

\section*{Exam 2 Material:}
The following material will mark the beginning of exam 2 material for the course.
Mixtures WILL NOT be on exam 1. 

\section*{Mixtures}
For this course we will only discuss binary mixtures (mixtures between two different chemical species).
Tertiary (and higher) mixtures are more complicated, and require more math, but no new concepts. 

x$_i$ defines the mole fraction of a species. 
If you are working with a closed system, the sum over all of the mole fractions must add up to 1. 
For a binary mixture of species A and B this implies. 
\be
\begin{split}
\sum_i x_i = 1\\
\rightarrow x_A + x_B = 1\\
\end{split}
\ee

When studying mixtures it becomes useful to discuss partial molar quantities (quantities in terms of the individual species in the mixture). 
The partial molar volume is defined as
\be
V_i = \left(\frac{\pd V}{\pd n_i}\right)_{P,T, n_j}
\ee
When you add a mole of substance A to a mixture, the volume change you measure would be the molar volume of A. 

This is not a simple linear function, if A does not like B, you would expect the change in molar volume to be different than if A does like B. 
In general you expect molecules that have an unfavorable interaction to want to stay separated (causing a larger increase in volume) whereas molecules with a favorable interaction will be happy to get close. 

We should also guess that partial molar volume is sensitive to concentration. 
If you add a single atom of molecule A to a solution of B, the A is completely surrounded by B. 
If there is already some A in solution, the A molecules may be able to pair up.
If A and B have very different interactions this will have a huge impact on the molar volume as you change the concentration. 
This means the slope of volume versus moles will most probably not be a linear function. 

For example consider water and ethanol.
We know water is a bit special, its hydrogen bond network actually forces liquid water to have a larger density than solid water (ice floats). 
If we consider mixing a small amount of water into ethanol there will be very few hydrogen bonds (ethanol and water cannot hydrogen bond). 
As the water concentration increases more hydrogen bonds can form, and we would guess the volume of the solution would increase rapidly. 

Mathematically speaking we are considering the volume to be a function of the chemical species in solution. 
This means we can write down a total differential with respect to composition. 
\be
\begin{split}
dV &= \left(\frac{\pd V}{\pd n_A}\right)_{P,T, n_B} dn_A + \left(\frac{\pd V}{\pd n_B}\right)_{P,T, n_A} dn_B \\
dV &\equiv V_Adn_A + V_B dn_B \\
V &= \int_0^{n_A} V_Adn_A + \int_0^{n_B} V_B dn_B \\
V &= V_A n_A + V_B n_B
\end{split}
\ee

\subsection*{Chemical Potential}
As we discussed previously, the chemical potential turns out to be a useful quantity to discuss in the context of mixtures. 
For a multi-component system we generalize our definition of the chemical potential too:
\be
\mu_i \equiv \left(\frac{\pd G}{\pd n_i}\right)_{P,T,n_j}
\ee
With this definition we can treat the operators like algebra and write
\be
\begin{split} \label{GD}
dG &= \sum_i \mu_i dn_i\\
G &= \int \sum_i \mu_i dn_i\\
G &= \sum_i \mu_i n_i
\end{split}
\ee
This statement is only true at constant P,T, n$_{i\ne j}$.
If we allow all of these variables to change we must generalize our definition of the Gibbs Free Energy. 
\be
\begin{split}
& \qquad \qquad G(T,P,n_i) \rightarrow \\
dG &= \left(\frac{\pd G}{\pd T}\right)_{P,n_i} dT + \left(\frac{\pd G}{\pd P}\right)_{T,n_i} dP+ \sum_i \left(\frac{\pd G}{\pd n_i}\right)_{T,P,n_j} dn_i  
\end{split}
\ee

It is important to realize that EVERY thermodynamic potential must be a function of composition (they are all extensive). 
Taking this general definition of G, you can go back through and substitute it into all of the other thermodynamic potentials.
The chemical potential term will simply be added on to each expression when you do the algebra. 

Introducing the chemical potential in this way allows us to generalize all of our fundamental equations, now in terms of multi-component systems. 
\begin{equation}
\begin{split}
    dU &= TdS - PdV + \sum_i\mu_i dn_i \implies U\rightarrow U(S,V,n) \\
    dH &= TdS + VdP + \sum_i\mu_i dn_i \implies H\rightarrow H(S,P,n)\\
    dA &= -SdT - PdV + \sum_i\mu_i dn_i \implies U\rightarrow A(T,V,n)\\
    dG &= -SdT + VdP + \sum_i\mu_i dn_i \implies G\rightarrow G(T,P,n)
\end{split}
\end{equation}

It is important to realize that EVERY thermodynamic potential must be a function of composition (they are all extensive). 

\subsection*{Gibbs-Duhem Equation}
Consider our 2 component system, a change in the Gibbs Free Energy then becomes
\be
dG = VdP - SdT + \mu_Adn_A + \mu_B dn_B
\ee

If we assume constant P and T this implies
\be
dG = \mu_A dn_A + \mu_B dn_B
\ee

Now consider Equation \ref{GD} again. 
We could write down the two component system as.
\be
\begin{split}
G &= \mu_A n_A + \mu_B n_B\\
dG &= \mu_A dn_A + n_A d\mu_A + \mu_B dn_B + n_B d\mu_B
\end{split}
\ee
We know thermodynamics must be consistent, this imples equating these two expressions for dG. 
\be
\begin{split}
\mu_A dn_A + \mu_B dn_B &= \mu_A dn_A + n_A d\mu_A + \mu_B dn_B + n_B d\mu_B \\
0 &= n_A d\mu_A + n_B d\mu_B\\
d\mu_B &= -\frac{n_A}{n_B}d\mu_A
\end{split}
\ee
This final result is known as the \textbf{Gibbs Duhem Equation} (for a two-component system). 
This equation essentially tells us that the chemical potential of two different species are proportional (wrt the ratio of their amounts). 
This should make sense, if we change one component of the system it directly impacts the other component. 

This equation is actually extremely useful. 
In complicated mixing problems (more than 2 components), the Gibbs-Duhem equation relates the species through their chemical potentials. 
This means all of the components ARE NOT independent, and you can actually reduce the number of variables you need to describe the system. 
A piece of advice, if you are dealing with a mixing problem write down the Gibbs-Duhem relationship, it can probably help you solve the problem (q\_q)

\end{document}