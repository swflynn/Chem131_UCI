\documentclass{article}
\usepackage[utf8]{inputenc}

\title{Chem132A: Lecture 09}
\author{Shane Flynn (swflynn@uci.edu) }
\date{10/18/17}


\usepackage{graphicx}
\usepackage{amsmath}
\usepackage{braket}
\usepackage[margin=0.7in]{geometry}
\usepackage{hyperref}


\newcommand{\be}{\begin{equation}}
\newcommand{\ee}{\end{equation}}
\newcommand{\benum}{\begin{enumerate}}
\newcommand{\eenum}{\end{enumerate}}
\newcommand{\pd}{\partial}

\begin{document}

\maketitle

This is the LAST LECTURE for new material that will be covered on EXAM 1. 
Any new material discussed in the course before Exam 1, will not be tested until Exam 2!

\section*{Exam Logistics}
The first exam will be Wednesday October 25!
The exam will be held in class and will last 45 minutes (the last 5 minutes of lecture will be used to collect the exam). 
Students are expected to show up to the exam ON TIME. 
Please bring your student I.D. and a calculator. 
Finally, a seating chart will be made available the day before the exam.
You MUST look at the seating chart before taking the exam and sit in the correct place!

The exam is closed book, you will be allowed to bring in a single 8.5 X 11 inch. sheet of paper with HAND-WRITTEN notes (both front and back allowed). 
You may write whatever you like on the piece of paper. 

Constants associated with the problems will be provided on the exam, however, no equations will be provided unless necessary (such as a unique equation of state we want you to use).
You must show ALL your work on the exam, simple numerical answers or algebraic expressions without associated work will earn NO points. 

\section*{Phase Boundaries}
Let's consider how the chemical potential varies with pressure at constant temperature. 
\be
dG = VdP - SdT = VdP
\ee
Rearranging we find that
\be
\left( \frac{\pd G}{\pd P} \right)_T = V
\ee
If we simply divide through by the moles we could write this relationship as a molar quantity.
The molar volume, by definition, must be larger than 0 (what is a negative volume? it has no physical meaning). 
From intuition we would expect the molar volume of a liquid to be larger than a solid, however, in water this is NOT true, due to the Hydrogen bond network. 

As stated in the previous lecture, at equilibrium the chemical potential for the phases in equilibrium must be the same. 
\be
\mu(\alpha) = \mu(\beta)
\ee
This means that the chemical potential on the phase line between phases must be equal for the two phases. 

If we consider our definition of the chemical potential for a single component system as $\mu$ = G, we can write the following relationship. 
Start with the fundamental equation for Gibbs on a phase line, but replace G with $\mu$ (we will consider molar quantities for simplicity). 
\be
\begin{split}
d\mu &= -S_mdT + V_mdP\\
-S_m(\alpha)dT + V_m(\alpha)dP &= -S_m(\beta)dT + V_m(\beta)dP\\
\left[ S_m(\beta) - S_m(\alpha) \right]dT &= \left[ V_m(\beta) - V_m(\alpha) \right]dP \\
\Delta_{\text{trs}}S_mdT  &=  \Delta_{\text{trs}}V_mdP \\
\frac{dP}{dT} &= \frac{\Delta_{\text{trs}} S}{\Delta_{\text{trs}} V}
\end{split}
\ee
This final equation is known as the \textbf{Clapeyron Equation}.

\subsection*{Liquid-Vapour Boundary}
Let's apply our new equation to the Liquid-Vapour Boundary. 
\be
\Delta S_{\text{vap}} = \left( \frac{\Delta H_{\text{vap}}}{T} \right)
\ee
With this we can substitute into the Clapeyron Equation to find
\be
\frac{dP}{dT} = \frac{\Delta H_{\text{vap}}}{T\Delta V_{\text{trs}}}
\ee
If we now assume that V$_m$(g) $>>$ V$_m$(l) we could say $\Delta_{\text{trs}}$V $\approx$ V$_m$(g). 
If we then assume an equation of state like the ideal gas we can solve for V$_m$
Applying this logic we find
\be
\frac{dP}{dT} = \frac{\Delta H_{\text{vap}}}{T(RT/P)} =\frac{P\Delta H_{\text{vap}}}{RT^2}
\ee
Math aside, $\frac{d}{dx} \ln x$ = $\frac{1}{x}$. This is a very common substitution people use. 
If we rewrite our LHS as ln P, than our RHS would need to be multiplied by $\frac{1}{P}$. 
With this logic we can simplify the expression to give the \textbf{Clausius-Clapeyron Equation}.
\be
\frac{d\ln P}{dT} = \frac{\Delta H_{\text{vap}}}{RT^2}
\ee
Finally we can assume that the Enthalpy of Vaporization is independent of temperature and integrate the Clausius-Clapeyron Equation.
\be
\begin{split}
\int_{\ln P^*}^{\ln P} \ln P &= \frac{\Delta H_{\text{vap}}}{R} \int_{T^*}^T \frac{dT}{T}\\
\ln \left( \frac{P}{P^*} \right) &=  -\frac{\Delta H_{\text{vap}}}{R} \left( \frac{1}{T} - \frac{1}{T^*} \right)
\end{split}
\ee
This last equation gives us a relationship regarding the change in vapour pressure wrt. T for an ideal gas. 
Where the vapour pressure is the pressure of gas above a liquid when the gas and liquid are at equilibrium. 

\end{document}