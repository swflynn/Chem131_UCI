\documentclass{article}
% Packages
\usepackage[utf8]{inputenc}
\usepackage{graphicx}
\usepackage{amsmath}
\usepackage{braket}
\usepackage[margin=0.7in]{geometry}
\usepackage{hyperref}
\usepackage[version=4]{mhchem}
% User-Defined Commands
\newcommand{\be}{\begin{equation}}
\newcommand{\ee}{\end{equation}}
\newcommand{\benum}{\begin{enumerate}}
\newcommand{\eenum}{\end{enumerate}}
\newcommand{\pd}{\partial}
% Title Information
\title{Chem132A: Lecture 17}
\author{Shane Flynn (swflynn@uci.edu)}
\date{11/15/17}

\begin{document}
\maketitle

\section*{Flux}
We ended last lecture by claiming that our model of an 'ideal gas' for chemical reactions, defines the gas atoms to be hard spheres.
There spheres randomly travel in straight lines in space and have elastic collisions. 

The \textbf{Flux} measures the amount of 'stuff' passing through a given area in a given amount of time. 

The flux of matter through a given area is referred to \textbf{Diffusion}.
Flux does not need to be matter, you can measure the flux of any 'thing' through an area in time.

\subsection*{Fick's Law}
Empirically the flux is usually approximated as the spatial derivative of some property.
For example \textbf{Fick's First Law} (of diffusion) is given by
\be
J(matter) \approx \frac{d}{dz}N
\ee
If you want, we can convert this to  a single spatial dimension, and consider mass transport in terms of concentration of species i; [i].
\be
Flux \approx \frac{d}{dx}[i]
\ee
We should note; by construction the flux is an average quantity, its dimensions are $\frac{\text{property}}{\text{distance}^2 \text{time}}$. 
The net flux is a function of concentration, things tend to go from high concentration to low concentration via diffusion. 

We can make the approximate statement above an equation using a linear approximation 
\be
J(matter) = -D\frac{d}{dz}N
\ee
Where D is the \textbf{Diffusion Coefficient}.
Consider the dimensions of this equation.
Given our units for the flux, the diffusion coefficient must have units of $\frac{m^2}{s}$.
While these units are correct, they may not seem intuitive!

We could also consider the movement of energy down a Temperature gradient and define a thermal motion diffusion equation. 
\be
J(thermal \quad energy) = -\kappa \frac{d}{dz}T
\ee
Here $\kappa$ is know as the \textbf{Thermal Conductivity}, and has units of $\frac{W}{m-K}$. 
Thermal transport is important in experimental work.
When you heat something in a laboratory, you are usually using a heating block, and this means the heat must travel throughout the system, this is dictated by the thermal flux. 

\subsection*{Viscosity}
Another transport property of interest is the viscosity. 
We know from intuition that viscosity measures the stickiness' of a solution, how hard it is to move through it (i.e. syrup is more viscous than water). 
Viscosity in the context of transport measures the transfer of linear momentum in your system. 
\be
J(momentum) = -\eta \frac{d v_x}{dz}
\ee
Where $\eta$ is the \textbf{Coefficient of Viscosity}. 

We can think of viscosity as the flux of linear momentum. 
The idea is that a fluid is flowing through some region of space.
If you are close to a stationary wall, than static friction exists between the wall and the closest fluid, which causes the layers close to the wall to have a lower rate. 
When you move between a region of low momentum to a region of higher momentum you must conserve energy.

\end{document}