\documentclass{article}
% Packages
\usepackage[utf8]{inputenc}
\usepackage{graphicx}
\usepackage{amsmath}
\usepackage{braket}
\usepackage[margin=0.7in]{geometry}
\usepackage{hyperref}
\usepackage[version=4]{mhchem}
% User-Defined Commands
\newcommand{\be}{\begin{equation}}
\newcommand{\ee}{\end{equation}}
\newcommand{\benum}{\begin{enumerate}}
\newcommand{\eenum}{\end{enumerate}}
\newcommand{\pd}{\partial}
% Title Information
\title{Chem132A: Lecture 16}
\author{Shane Flynn (swflynn@uci.edu)}
\date{11/6/17}

\begin{document}
\maketitle

\section*{Logistics}
There will be no lecture this Friday (11/10/17) due to Veterans Day. 
There will also be no lecture the following Monday (11/13/17) due to Professor Hemminger traveling. 

Make sure you check your Exam 1 score to see if there were any grading mistakes.
There are no re-grades for partial credit.
If the grade was added incorrectly, or an answer was marked wrong (but is correct) please contact the TAs or Professor. 
The Second Midterm will be Wednesday 11/22/17 (the day before Thanksgiving). 

Please note: all exams must be taken unless an excused absence (with supporting documentation i.e. sickness) is granted.
If a student misses an exam without any excused reason, they will receive a 0 on the exam, and they will not be allowed to 'drop' this exam.
Although Professor Hemminger will re-calculate your grade with and without your lowest score, if you receive a 0 on an exam without proper reason, you \textbf{Will Not} be allowed to drop the score (i.e. your grade \textbf{WILL NOT} be calculated with one dropped exam). 

Exam 2 will cover Chapters 1-6 and Chapter 19. 
The exam is cumulative by nature, expect at-least one question to come from material that was covered  before Exam 1.

\subsection*{Ionic Solutions}
The Debye-Huckel Limiting Law would predict the log  $\gamma_i$ to be linearly proportional to the square root of ionic  strength.
This is not always true, as shown in the Lecture figure, which suggests the Limiting Law is not a perfect representation of ionic solutions, only an approximation. 

\subsection*{Phase Rule}
A final comment, we discussed the Gibbs Phase Rule  when introducing phase diagrams.
\be
F = C - P + 2
\ee
Where the Degrees of Freedom (F) is equal to the number of components (C) minus the number of phases (P) minus 2. 
For a 2-component system we find that F = 4-P.
These phase diagrams are not as simple as before, and the Gibbs Phase Rule can help you understand the chemical species present at different conditions. 

\section*{Chemical Equilibria}
Chapter 6 is the last chapter of standard Thermodynamics covered in this course (we will study Chemical Kinetics next). 

As we started discussing last lecture, the mixing effects associated with the extent of reaction completely change the nature of the Thermodynamic Potentials. 
For an ideal system the Enthalpy of Mixing is 0, therefore the Gibbs Free Energy of Mixing must depend solely on the Entropy. 
However, in a non-ideal scenario the Enthalpy of mixing is not necessarily zero. 

\subsection*{Equilibrium Constant}
In General Chemistry you were taught the Equilibrium Coefficient should be the ratio of products over reactants (concentration).
It would be more accurate to define the ratio in terms of activity.
Consider the chemical reaction $\ce{A + B <=> C + D}$
\be
K \equiv \frac{a_Ca_D}{a_Aa_B}
\ee
We can relate the activity to amount, if we wish to use mole fraction we would substitute in $a_i = \gamma_i \frac{b_i}{b^o}$. 
Substituting in this equation we find
\be
K = \frac{\gamma_C\gamma_D}{\gamma_A\gamma_B}\frac{b_Cb_D}{b_Ab_B} \equiv K_\gamma K_b
\ee

\subsection*{Temperature Dependence of K}
At equilibrium we were able to write
\be
\Delta_rG^o = -RT\ln K
\ee
If we are interested in the Temperature dependence of K, we can simply re-arrange this equation and take a derivative wrt. T. 
\be
\begin{split}
\ln K = -\frac{\Delta_rG^o}{RT} \implies& \frac{d}{dT}\left(\ln K\right) = -\frac{1}{R} \frac{d}{dT}\left(\frac{\Delta_rG^o}{T}\right)\\
&= -\frac{1}{R} \frac{d}{dT}\left(\frac{\Delta_rH^o - T\Delta_rS^o}{T}\right)\\
\frac{d}{dT} \ln K & = \frac{\Delta_rH^o}{RT^2}
\end{split}
\ee
Note: in this derivation we are taking derivatives of reference states, we assume these reference states are not functions of temperature. 

Using this result we can make a simple substitution
\be
\frac{d}{dT}\left(\frac{1}{T}\right) = -\frac{1}{T^2}
\ee
Using this substitution we can express the last line  of the previous derivation as
\be
\frac{d}{d (1/T)}\ln (K) = -\frac{\Delta_rH^o}{R}
\ee
Finally if we integrate this last expression we find
\be
\ln K_2 - \ln K_1 = -\frac{\Delta_rH^o}{R}\left(\frac{1}{T_2} - \frac{1}{T_1}\right)
\ee
(See the discussion homework for a problem regarding this derivation).

\section*{The Rest of Chapter 6 $_{\cdots}$}
The remainder of the chapter covers topics that should have been taught in General Chemistry (Le Chatelier's Principle, and Electro-Chemistry). 
This material will not be covered in lecture, however, it is fair game for exams and is part of this course.
If you are unfamiliar with any of the topics covered in these sections please review them (potentially reference your General Chemistry Textbook). 

This is the end of Thermodynamics, the remainder of the course will shift gears towards Chemical Kinetics.
We will start with a discussion of Chapter 19 next lecture (Molecules in Motion). 


\end{document}