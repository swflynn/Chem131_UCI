\documentclass{article}
\usepackage[utf8]{inputenc}

\title{Chem132A: Lecture 0}
\author{Shane Flynn (swflynn@uci.edu) }
\date{9/29/17}


\usepackage{graphicx}
\usepackage{amsmath}
\usepackage{braket}
\usepackage[margin=0.7in]{geometry}
\usepackage{hyperref}


\newcommand{\be}{\begin{equation}}
\newcommand{\ee}{\end{equation}}
\newcommand{\pd}{\partial}

\begin{document}

\maketitle

\section*{Chem132A Welcome and Logistics}
Welcome to Thermodynamics!

There is a course website on canvas,
\url{https://canvas.eee.uci.edu/courses/6058}
However, this github should essentially be self-contained. 

The textbook for the course is: Physical Chemistry; Thermodynamics, Structure, and Change (10 Edition) by Peter Atkins and Julio de Paula. 
\textbf{Webassign} (online homework) associated with the book will be required for the course and needs to be purchased by students too. 
For this reason you must be \textbf{registered} in the course (speak to professor Hemminger if you are not registered). 

The course is taught by powerpoint, and a pdf of the lecture will be provided after the lecture. 
These lecture notes are intended to be supplemental information associated with the powerpoint. 

\subsection*{Discussions}
There will be 8 discussion sections each week, and students are required to be enrolled in one. 
It does not actually matter what section you attend, just make sure you are enrolled in one. 
Discussions are \textbf{NOT} mandatory.
We encourage students to attend, however, there will be no attendance or credit earned for attending. 

\section*{Grades}
The course breakdown will be something like: 
\begin{enumerate}
\item Webassign; 5\% (Each Week)
\item Midterm 1; 20\% (October 25, Wednesday)
\item Midterm 2; 20\% (November 22, Wednesday)
\item Final; 55\% (December 15, Friday 8am-10am)
\end{enumerate}

No make-up exams will be provided to students. 
You need to make every effort to attend each exam!
If for some reason you will need to miss an exam (and you know in advance) speak to Professor Hemminger ASAP.

Obviously things come up in life, if for some reason you miss an exam, with a valid reason for doing so, speak to Professor Hemminger (and be prepared to provide documentation and etc). 

\section*{Getting Started}
Chapter 1 of the textbook will be assumed to be background material.
Students should read this chapter, and there is an associated webassign homework (not for a grade) addressing these questions.
The first discussion problem set will also discuss the contents of Chapter 1. 

The course as a whole will cover the fundamentals of thermodynamics, and end with a discussion of chemical kinetics. 

We will start with a discussion of the Ideal Gas Law, and slowly develop an intuition for how real chemical systems work. 



\end{document}