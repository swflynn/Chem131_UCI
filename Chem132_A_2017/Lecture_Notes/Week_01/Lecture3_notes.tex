\documentclass{article}
\usepackage[utf8]{inputenc}

\title{Chem132A: Lecture 03}
\author{Shane Flynn (swflynn@uci.edu) }
\date{10/4/17}


\usepackage{graphicx}
\usepackage{amsmath}
\usepackage{braket}
\usepackage[margin=0.7in]{geometry}
\usepackage{hyperref}


\newcommand{\be}{\begin{equation}}
\newcommand{\ee}{\end{equation}}
\newcommand{\benum}{\begin{enumerate}}
\newcommand{\eenum}{\end{enumerate}}
\newcommand{\pd}{\partial}

\begin{document}

\maketitle

\section*{Summary Previous Lecture}
During the previous Lecture we discussed the First Law of Thermodynamics; The Internal Energy of an isolated system is constant.
This Law is commonly referred to as the conservation of energy. 

It is important to understand that Internal Energy (U) is a state function, but the heat (q) and work (w) are not.
We then discussed various assumptions one can make to simplify the equations for work. 

At the very end of lecture we gave a name to a new partial derivitive we encountered, the \textbf{Constant Volume Heat Capacity}. 
\be
C_V \equiv \left(\frac{\pd U}{\pd T}\right)V
\ee
Intuitively the heat capacity measures the change in energy experienced by a material with an associated change in temperature. 

\section*{Heat Capacity}
We can motivate the definition of the constant volume heat capacity as follows.
Because the Internal Energy is a state function we can write a total differential for it. 
If we consider U(T,V) the total differential would be 
\be
dU = \left(\frac{\pd U}{\pd T}\right)_VdT + \left(\frac{\pd U}{\pd V}\right)_T dV 
\ee
If we restrict our system to a constant volume system (no PV work), than this total differential becomes 
\be
dU = \left(\frac{\pd U}{\pd T}\right)_VdT \equiv C_v dT
\ee
With this definition we can assume a heat capacity that is independent of temperature to simplify the integral. 
\be
\Delta U = \int_{T_i}^{T_f} C_V(T) dT = \int_{T_i}^{T_f} C_V dT = C_v\Delta T
\ee
Finally if we are assuming a process done at constant volume and no other work besides PV work the First Law will simplify to $\Delta U = q$ and we can write
\be
\Delta U = q = C_v\Delta T 
\ee

\section*{Enthalpy}
We have seen that the Internal Energy can be used to define useful properties describing a system. 
We are now going to introduce an new variable called the \textbf{Enthalpy} (H). 
For this course we will take the following equation to be a definition. 
\be
H \equiv U + PV
\ee
From inspection we see that U, P, and V are all state functions. 
It turns out that a function composed of state functions is also a state function.

Meaning H is another state function!
\textbf{Comment:} This logic is not the entire story. 
For example we know U = q + w, and neither q nor w are state functions, and yet U is a state function. 
The reason for this is beyond an undergraduate course, but it is interesting to think about. 

To get familiar with this new equation, consider the enthalpy change for a reaction between ideal gases. 
The Enthalpy can be expressed as using the ideal gas law as
\be
\Delta H = \Delta U + \Delta n_g RT
\ee

Let's explore this new equation some more. 
Consider the following.
\be
\begin{split}
H = U + PV \Rightarrow dH &= dU + d(PV) \Rightarrow dH = dU + PdV + VdP \\
dH &= \delta q + \delta w + PdV + VdP\\
\end{split}
\ee
We have substituted in the First Law, if we consider a process with only PV work we can simplify the expression some more.
\be
\begin{split}
dH &= \delta q + \delta w + PdV + VdP\\
dH &= \delta q + -PdV + PdV + VdP\\
dH &= \delta q + VdP\\
\end{split}
\ee
Finally if I assume that the process occurs at a constant pressure we find an interesting relationship. 
\be
\begin{split}
dH &= \delta q + VdP\\
dH &= \delta q_P \Rightarrow \Delta H = q_P \\
\end{split}
\ee
This result says, given a certain set of assumptions we are able to directly relate heat to the Enthalpy. 

Similar to the process we did with U, consider H(T,P), the total differential would be. 
\be
dH = \left(\frac{\pd H}{\pd T}\right)_PdT + \left(\frac{\pd H}{\pd P}\right)_T dP
\ee
If we consider a constant pressure process we can define a new heat capacity, at constant pressure this time. 
\be
dH = \left(\frac{\pd H}{\pd T}\right)_PdT = C_pdT
\ee
Where we now define the constant pressure heat capacity from the total differential as
\be
C_p \equiv \left(\frac{\pd H}{\pd T}\right)_PdT 
\ee
Again we can assume the heat capacity to be independent of temperature and we find the following. 
\be
\begin{split}
dH &= C_PdT \Rightarrow \\
\Delta H = \int_{T_i}^{T_f} C_P(T) dT &= \int_{T_i}^{T_f} C_P dT = C_P\Delta T
\end{split}
\ee

It would seem that the enthalpy can be useful during a constant pressure process, and the internal energy can be used in a constant volume process. 
In Thermodynamics we can define any number of equations, it is important to understand why a certain definition may be useful!

\section*{Thermochemistry}
During general chemistry you should have discussed the Enthalpy, and in general the concept of \textbf{Thermochemistry}. 
Section 2C in Atkins covers the basics of Thermochemistry, and this information is assumed to be background knowledge for this course. 
If you are unfamiliar with any of the topics in this section please review them on your own. 

This section covers topic like the enthalpy of reaction ($\Delta_rH $) and the enthalpy of formation ($\Delta_fH $), here $\nu$ defines the stoichometric coefficient for the species. 
\be
\Delta_rH = \sum_{\text{product}}\nu \Delta_fH - \sum_{\text{react}}\nu\Delta_fH
\ee

If we are interested in using tabulated values for the enthalpy to study a reaction we may need to convert between temperatures.
This is done easily by evaluating the enthalpy at the know temperature and the constant pressure heat capacity. 
\be
H(T_2) = H(T_1) + \int_{T_1}^{T_2} C_p(T) dT
\ee

In the same manner we can construct the enthalpy of reaction at different temperatures by substituting the appropriate terms. 
\be
\Delta_rH(T_2) = \Delta_rH(T_1) + \int_{T_1}^{T_2} \Delta_rC_P(T) dT
\ee
With the analogous definition for determining heat capacities.
\be
\Delta_rC_P(T) = \sum_{\text{product}}\nu  C_P - \sum_{\text{react}}\nu C_P
\ee

In the future we will compute the enthalpy of a reaction that undergoes a phase change.
As you can probably guess, you simply add in a term to account for the enthalpy change during the transition itself. 
As stated above, this section is meant to be a review, please read the book if this summary is not detailed enough.

\section*{Naming Partial Derivatives}
We have already seen an example where we give an important partial derivative a name (heat capacities). 
There are various other derivatives that can be written down and calculated using thermodynamics. 
Engineers may even be able to measure these properties experimentally and tabulate them for you to use. 

We can write a total differential down for any state function, where we simply consider a summation of all of the partial derivatives composing the state function. 
We can write this summation in any order, because none of the variables depend on their path (meaning they do not depend on their order either).
If we have an equation of state we can actually reduce the number of variables we need to define our state function.
For example: Using PV=nRT, we only need to define 3 variables, and we can solve for the fourth. 
It does not matter which 3 we specify, we can always determine the remaining variable, meaning the variables are not all independent. 

Consider again the total differential of U(T,V). 
\be
dU = \left(\frac{\pd U}{\pd T}\right)_VdT + \left(\frac{\pd U}{\pd V}\right)_T dV \equiv C_VdT + \pi_TdV 
\ee
We mentioned the heat capacity above, but consider our new definition, the \textbf{Internal Pressure}
\be
\pi_T \equiv \left(\frac{\pd U}{\pd V}\right)_T dV 
\ee
The internal pressure characterizes the change in energy associated with a change in volume (at constant temperature). 

In general we can write U(P,V,T) (in chemistry we usually consider closed systems so we will set n=1 and ignore it). 
As stated above, we can use an equation of state to reduce how many variables we need to characterize the system.
Meaning we could write U(P,V), U(P,T), U(T,V), all of which give the same information. 

As you can guess, we can also write down various partial derivatives associated with these manipulations. 

If we wanted to see how U varies as a function of T we could rearrange our total differential (divide by dT holding P constant). 
\be
\left(\frac{\pd U}{\pd T}\right)_P = \pi_T\left(\frac{\pd V}{\pd T}\right)_P + C_V
\ee
This manipulation gives a new derivative, which we can define to be the \textbf{Expansion Coefficient}, the change in volume with an associated change in temperature. 
\be
\alpha = \frac{1}{V} \left(\frac{\pd V}{\pd T}\right)_P
\ee
Where we normalize by volume to make comparisons between materials.

\end{document}