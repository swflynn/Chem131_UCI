\documentclass{article}
\usepackage[utf8]{inputenc}

\title{Chem132A: Lecture 07}
\author{Shane Flynn (swflynn@uci.edu) }
\date{10/13/17}


\usepackage{graphicx}
\usepackage{amsmath}
\usepackage{braket}
\usepackage[margin=0.7in]{geometry}
\usepackage{hyperref}


\newcommand{\be}{\begin{equation}}
\newcommand{\ee}{\end{equation}}
\newcommand{\benum}{\begin{enumerate}}
\newcommand{\eenum}{\end{enumerate}}
\newcommand{\pd}{\partial}

\begin{document}

\maketitle


The first midterm will cover Chapters 2-4 in the book. 
The reason for having the exams in the course is to ensure students have some feedback as to where they are in the class.
Thermodynamics is a challenging subject for many students, it is important that you stay up-to-date with the material. 

\section*{Math Lesson}
In the Week 3 Discussion homework, you will be asked to derive all of the fundamental equations and all of the Maxwell Relationships.
See the solutions to that homework for the algebra. 

Consider the Internal Energy Fundamental Equation
\be
dU = TdS - PdV
\ee
This tells us a change iin U is associated iwth changes in S and V.
We can therefore construct a total differential of U (because it is a state function) as
\be
dU = \left(\frac{\pd U}{\pd S}\right)_V dS + \left(\frac{\pd U}{\pd V}\right)_S dV
\ee

\subsection*{Exact Differentials}
An exact differential id a special type of function where the following relationship is true.
\be
    \frac{\partial}{\partial y} \left[\left(\frac{\partial f}{\partial x}\right)_y \right]_x =  \frac{\partial}{\partial x} \left[\left(\frac{\partial f}{\partial y}\right)_x \right]_y
\ee
Physically speaking this result says the order in which we take our partial derivatives does not matter. 
This can be interpreted as that pathway in which the process is completed does not matter, which is exactly true for state function. 

\subsubsection*{Quick Example}
Consider f(x,y) = x$^2$y$^3$, this is a 'separable' function and the second order derivatives will be exact. 
Let's just plug this into our definition above and show this relationship is true. 
\be
\begin{split}
    \left(\frac{\partial f}{\partial x}\right)_y = 2xy^3 \\
    \left(\frac{\partial}{\partial y}\right)\left[ \left(\frac{\partial f}{\partial x}\right)_y\right]_x = 6xy^2
    \end{split}
\ee
Now consider the reverse order of the partials
\be
\begin{split}
    \left(\frac{\partial f}{\partial y}\right)_x = 3x^2y^2 \\
    \left(\frac{\partial}{\partial x}\right)\left[ \left(\frac{\partial f}{\partial y}\right)_x\right]_y = 6xy^2
    \end{split}
\ee

This relationship is only true under special conditions, relating to thermodynamics we know the path functions such as heat and work depend on the order.
We can therefore NOT write an exact differential for a path function. 

\bigskip

\textbf{Comment:}
Please review the solutions to the third Discussion homework for how to derive all of these relationships wrt. the four Thermodynamic Potentials (U,S,A,G). 

\section*{Gibbs Free Energy and Temperature}
The Gibbs Free Energy is useful, because the characteristic variables are Temperature and Pressure (things we can easily control). 

The fundamental Equation for Gibbs turns out to be 
\be
dG = VdP - SdT 
\ee
We also know the definition of G = H - TS therefore S = (H-G)/T.
The fundamental Equation tells us that
\be
\left(\frac{\pd G}{\pd T}\right)_P = -S = \frac{G-H}{T}
\ee
Now consider a product rule of the derivative of G/T. 
\be
\begin{split}
\left(\frac{\pd \frac{G}{T}}{\pd T}\right)_P &= \frac{1}{T}\left(\frac{\pd G}{\pd T}\right)_P + G \left(\frac{\pd \frac{1}{T}}{\pd T}\right)_P\\
&= \frac{1}{T}\left[\left(\frac{\pd G}{\pd T}\right)_P - \frac{G}{T}\right]
\end{split}
\ee
We can simplify this last term in brackets by
\be
\left(\frac{\pd G}{\pd T}\right)_P - \frac{G}{T}= \frac{G-H}{T} - \frac{G}{T} = -\frac{H}{T}
\ee
Therefore the final relationship is that
\be
\left(\frac{\pd \frac{G}{T}}{\pd T}\right)_P = -\frac{dH}{T^2}
\ee
This relationship allows us to relate a change in Enthalpy to a change in the Gibbs Free energy (as temperature varies). 



\end{document}