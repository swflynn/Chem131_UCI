\documentclass{article}
\usepackage[utf8]{inputenc}

\title{Chem132A: Lecture 06}
\author{Shane Flynn (swflynn@uci.edu) }
\date{10/11/17}


\usepackage{graphicx}
\usepackage{amsmath}
\usepackage{braket}
\usepackage[margin=0.7in]{geometry}
\usepackage{hyperref}


\newcommand{\be}{\begin{equation}}
\newcommand{\ee}{\end{equation}}
\newcommand{\benum}{\begin{enumerate}}
\newcommand{\eenum}{\end{enumerate}}
\newcommand{\pd}{\partial}

\begin{document}

\maketitle

\section*{What We know}
We are going to take a minute and recap everything we have covered so far in the course. 
We began our discussion with the First Law of Thermodynamics, which is concerned with heat, work, and the Internal Energy. 
\be
\Delta U = q + w
\ee

We proposed that the internal energy is a state function, which can be very useful for solving problems. 
We also introduced the concept of equations of state, which are equations relating state variables. 
\be
PV = nRT
\ee

To evaluate the first law, we developed the concept of work.
In chemistry we usually only consider PV work which we can write as
\be
w_{PV} = -\int_{V_i}^{V_f}P_{ext}dV
\ee
We can then make assumptions about a process and the equation of state to write explicit forms of the work done on/by a system. 

We did not really  developed a formal equation to explain heat, however it is intuitive that it is related to Temperature. 
In this way it seems natural to write U(T,V). 

Because U is a state function we decided to write a total differential
\be
dU = \left(\frac{\pd U}{\pd T}\right)_VdT + \left(\frac{\pd U}{\pd V}\right)_T dV 
\ee
Some of these partial derivatives seem interesting and we gave them names
\be
C_V \equiv \left(\frac{\pd U}{\pd T}\right)_V
\ee
With this definition and a few assumptions we were then able to write an equation for heat
\be
q = C_V\Delta T
\ee

In this way we decided the Internal Energy was a useful concept, we were able to say something about heat, and we decided to look for more. 

Next we defined a new equation, the Enthalpy. 
\be
H \equiv U + PV
\ee
With some more assumptions we decided to write the enthalpy as H(T,P), which developed a new total differential and some new definitions. 
\be
C_P \equiv \left(\frac{\pd H}{\pd T}\right)_P
\ee
And using a new set of assumptions we were able to express the heat in a different form. 
\be
q = C_P\Delta T
\ee

The first Law ultimately track heat and work, but we are also interested in the spontaneity of things. 
To get at this concept, we developed the Second Law of Thermodynamics. 
We gave an equation for defining entropy
\be
dS \equiv \frac{\delta q_r}{T} \Rightarrow \Delta S = \int_i^f \frac{\delta q_r}{T}
\ee
And we claimed without any justification that Entropy was a state function. 
We gave a few statements about the second law such as the total entropy of the universe is strictly greater than 0 during a spontaneous process

A more general statement regarding entropy would be the following
\be
dS \geq \frac{\delta q}{T}
\ee
We get the equality when a process is done reversibly, and irreversible process is inefficient and we get the inequality. 

\section*{More Thermodynamic Potentials}
It became clear that discussing the entropy can be tricky, you need to account for the entire universe to make accurate statements.
This can be confusing, so we are going to develop some new Thermodynamic Potentials (new equations like U and H). 

\subsection*{Helmholtz Free Energy}
Let's start again at the First Law, assuming a closed system with PV work only and constant volume. 

\begin{equation}
    dU = \delta q
\end{equation}
Let's incorporate our Second Law, by substituting in the First Law with our assumptions of constant volume and a closed system.
\be
\begin{split}
    dS &\geq \left(\frac{dU}{T}\right)_V\\
    TdS &\geq dU \\
    0 &\geq dU - TdS
    \end{split}
\ee
If the temperature is also constant we can pull it into the differential and write. 
\begin{equation}
d(U-TS) \leq 0
\end{equation}
Which is true for a constant T,V process.
This result is very useful, the process as a whole must be negative to be spontaneous (or else we break our inequality). 
The internal energy, entropy, and temperature in this equation only refer to the system and nothing else!

From this analysis is seems useful to define a new thermodynamic potential 
\begin{equation}
A \equiv U - TS
\end{equation}
A is known as the Helmholtz Free Energy (sometimes labeled as F), and a negative Helmholtz implies a spontaneous process, because it satisfies our inequality. 
Essentially this potential is useful because it 'tracks' the  necessary increase in the Entropy of the universe during a process through monitoring the environment. 
If we consider a constant temperature process we have
\begin{equation}
\Delta A = \Delta U - T\Delta S
\end{equation}
We know that $\Delta$A must be negative to be spontaneous and you can compensate through negative Internal Energy or positive Entropy. 
A negative $\Delta$A implies energy enters the environment which will increase its Entropy. 

If we do a process reversibly we get our equality
\begin{equation}
    \Delta S_r = \frac{q_r}{T}
\end{equation}
We can use this to consider the reversible Helmholtz Equation. 
\begin{equation}
    \begin{split}
        \Delta A_r &= \Delta U_r - T\Delta S_r \\
        &= \Delta U_r - T\frac{q_r}{T} \\
        &= \Delta U_r - q_r \\
        &= q_r + w_r - q_r \\
        \Delta A_r &= w_r
    \end{split}
\end{equation}
So we see the Helmholtz free energy with constant Temperature done reversibly is equal to the reversible work of the process. 
The reversible work is by definition the maximum work you can generate from a process, because it is completely efficient. 
For this reason the Helmholtz Free Energy is sometimes referred to as the work function (Arbeit in German translates to Work). 

To interpret the Helmholtz consider:
\be
\Delta A = \Delta U - T\Delta S
\ee
For A to be negative (and therefore spontaneous) our change in Internal Energy must account for the systems change in Entropy. 
If the Entropy of the system decreases, the Internal Energy must be negative.
A negative change in Internal Energy can be thought of as energy leaving the system.
This energy enters the environment, increasing the entropy of the environment, satisfying the Second Law. 
In this way we track the requirements placed by the Second Law, any excess free energy can be harnessed to do useful work, after satisfying the Second Law.

\subsubsection*{Gibbs Free Energy}
Experimentalists would really like a thermodynamic potential with characteristic variables T and P. 
This way working in the lab they could easily control the natural variables for the function. 

Consider a constant pressure process, and only PV work.  
\begin{equation}
    \delta q = dH 
\end{equation}
Assuming a constant temperature we can use our Second Law to write
\be
\begin{split}
dS \geq \frac{\delta q}{T} \Rightarrow dS - \frac{\delta q}{T} \geq 0\\
TdS - dH \geq 0 \Rightarrow d(H-TS) \leq 0
\end{split}
\ee

Just like with the Helmholtz Free Energy, we are able to write a new relationship invoking the entropy of the system.
\begin{equation}
    G \equiv H - TS ,\qquad dG \leq 0 \Rightarrow \text{spontaneous}
\end{equation}
So the Gibss Free energy is a natural variable of G(P,T) and is negative for a spontaneous process. 
Where again the direction of spontaneity is ultimately accounting for the Second Law. 

For chemists the Gibbs Free Energy is very convenient for analyzing constant T, P processes. 
A negative Gibbs implies a spontaneous reaction under these conditions. 
Knowing this, we can actually think of equilibrium in a chemical reaction occurring when the change in Gibbs is 0. 
This implies no spontaneous tendency for a reaction to move in either direction, which is what occurs at equilibrium.
This concept will appear again when we discuss the Chemical Potential ($\mu$).

A similar interpretation of the Gibbs Free Energy exists, such that The Gibbs can be interpreted as the maximum non-expansion work available to a system. 
If you assume constant Temperature, constant Pressure, and a reversible process you find (Page 135).
\be
\begin{split}
G &= H - TS\\
H &= U + PV \\
G &= U + PV - TS\\
U &= q + w \\
G &= q + w + PV - TS\\
\end{split}
\ee
Now let's assume an isothermal reversible process.
Let's take the work to be PV work plus 'other work' (where other means shaft work electric work, magnetic work, etc).
\be
\begin{split}
dG &= \delta q + \delta w + PdV + VdP - TdS\\
dG &= \delta q -PdV - VdP + \delta w_{non-PV}+ PdV + VdP - TdS\\
dG_r &= TdS_r -TdS + \delta w_{non-PV} \\
dG_r &= \delta w_{non-PV}\\
\end{split}
\ee
Just like Helmholtz, with a given set of assumptions the Gibbs Free Energy can be shown to be directly equal to the NON-expansion work occurring in a system. 

\end{document}