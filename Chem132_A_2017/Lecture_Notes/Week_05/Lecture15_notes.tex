\documentclass{article}
% Packages
\usepackage[utf8]{inputenc}
\usepackage{graphicx}
\usepackage{amsmath}
\usepackage{braket}
\usepackage[margin=0.7in]{geometry}
\usepackage{hyperref}
% User-Defined Commands
\newcommand{\be}{\begin{equation}}
\newcommand{\ee}{\end{equation}}
\newcommand{\benum}{\begin{enumerate}}
\newcommand{\eenum}{\end{enumerate}}
\newcommand{\pd}{\partial}
% Title Information
\title{Chem132A: Lecture 15}
\author{Shane Flynn (swflynn@uci.edu)}
\date{11/3/17}

\begin{document}
\maketitle

\section*{Activity Coefficient}
Last lecture we discussed real systems, defining the activity to replace mole fraction.
We define the activity coefficient out of convenience, it makes all of our equations easy to work with.
For example 
\be
\mu_A = \mu_A^* + RT\ln x_A + RT \ln \gamma_A
\ee
The first two terms are simply the results of an ideal assumption, and the last term captures our deviations from an ideal system.

We then discussed the special case of non-ideal systems, ionic solutions.
In an ionic solution we can model the activity coefficient using the Debye-Huckel Limiting Law.
A major component of this Law is the ionic strength, which captures how 'strong' an ion effects the system.
From intuition you would expect this to depend on the charge of the ion and the amount of ions present in solution. 
For a solution with two types of ions  we could write
\be
\begin{split}
I &= \frac{1}{2}\sum_i z_i^2 \left( \frac{b_i}{b^o}\right) \\
I &= \frac{1}{2}\frac{\left(b_+z_+^2 + b_-z_-^2\right)}{b^o}
\end{split}
\ee
Remember a solution must be electrically neutral (all the charges balance) the ions must separate into an electrically balanced ratio. 
Because each ion generates a coulomb force, we would guess that a solution with more ions would be more sensitive to assuming ideal behavior.
This assumption is not the entire story however, it really depends on the nature of the system.
The charge density and polarizibility of an ion greatly impact  how non-ideal a certain ion behaves. 
For example a larger ion may polarize more (the electrons can spread out more), which is not a behavior accounted for by ideal equations. 

\section*{Beyond Debye-Huckel}
The Debye Huckel Limiting Law is the simplest model for an ionic solution activity.
It is not extremely accurate for every system, but it is a good place to start.
The book discusses extensions of the Law, these \textbf{WILL NOT} be part of this course.

\section*{Chemical Equilibrium}
Changing gears we are now going to discuss the tendencies of chemical systems. 

Consider a simple reaction
\be
A \rightleftharpoons B
\ee
We can define a new variable, the \textbf{Extent of Reaction} ($\xi$).
The extent of reaction tracks the conversion of products  and reactants. 
For example, in a closed system of A and B with no other species present, the extent of reaction for each species is coupled. 
\be
dn_A = -d\xi, \qquad \qquad \qquad dn_B = d\xi
\ee
Which means that any amount of A lost must be converted into B. 

Because we are considering species in a system changing it should seem natural to discuss the Gibbs Free Energy (think back to phase diagrams). 
We can define the \textbf{Reaction Gibbs Energy} as 
\be
\Delta_rG = \left(\frac{\pd G}{\pd \xi}\right)_{P,T}
\ee
Because we are considering a system at constant Temperature and Pressure, the only variables changing are the amounts of the materials.
This means we can write G as a function of A and B
\be
dG = \,u_Adn_A + \mu_Bdn_B = -\mu_Ad\xi + \mu_Bd\xi = (\mu_B-\mu_A) d\xi
\ee
Pretending we can treat operators algebraically we then write
\be
\left(\frac{\pd G}{\pd \xi}\right)_{P,T} = \mu_B-\mu_A
\ee

Just as in the case of phase Diagrams, at equilibrium the Gibbs Free Energy must be minimized.
For a constant T,P system this  implies the net interchange of A and B must be the same (aka the chemical potential of A and B must be the same). 
Therefore at equilibrium $\Delta_r$G = 0.

Let's expand out our equation for Gibbs in terms  of Ideal Gases A and B.
\be
\begin{split}
\Delta_rG &= \mu_B - \mu_A \\
&= \mu_B^o + RT\ln\left(\frac{P_B}{P^o}\right) - \mu_A^o + RT\ln\left(\frac{P_A}{P^o}\right) \\
&= \Delta_rG^o + RT\ln\left(\frac{P_B}{P_A}\right)\\
\Delta_rG &= \Delta_rG^o + RT \ln(Q)
\end{split}
\ee

Where in the above derivation we  have defined a reference Gibbs Free Energy as $\Delta_rG^o \equiv \mu_b^o - \mu_A^o$, and we define the \textbf{Reaction Quotient} (Q) as
\be
Q = \frac{P_B}{P_A}
\ee

Because we know at equilibrium the Gibbs Free Energy is minimized we can define the \textbf{Equilibrium Constant} (K) as the reaction quotient taken at equilibrium and write
\be
\Delta_rG^o = -RT\ln(K)
\ee

It is important to note that the above analysis ignores  the effects  of the mixing process entirely.
The Gibbs Free Energy as a function of Extent of Reaction looks completely different if we include mixing effects (the Entropy of mixing is very important). 

\end{document}