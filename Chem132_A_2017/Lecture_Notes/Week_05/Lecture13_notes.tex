\documentclass{article}
% Packages
\usepackage[utf8]{inputenc}
\usepackage{graphicx}
\usepackage{amsmath}
\usepackage{braket}
\usepackage[margin=0.7in]{geometry}
\usepackage{hyperref}
% User-Defined Commands
\newcommand{\be}{\begin{equation}}
\newcommand{\ee}{\end{equation}}
\newcommand{\benum}{\begin{enumerate}}
\newcommand{\eenum}{\end{enumerate}}
\newcommand{\pd}{\partial}
% Title Information
\title{Chem132A: Lecture 13}
\author{Shane Flynn (swflynn@uci.edu)}
\date{10/30/17}

\begin{document}
\maketitle

\section*{Homework}
We strongly advise everyone to look at Exercises: 5c.3a, 5c.3b, 5c.4a, 5c.7a and Problems: 5c.5, 5c.7

The average on exam 1 was a 50.1 and the standard deviation was 12.

\subsection*{Section 5D}
In the textbook, section 5D covers tertiary mixtures.
This topic is more mathematically challenging and we will be skipping the process. 
The results are more general, but not worth the effort to derive them in an undergraduate course. 

\subsection*{Phase Diagrams}
The book covers various phase diagrams associated with binary mixtures.
It is important to take a look at the graphs and understand what they are potting, and what information you gain from that graph. 
The four most common types of phase diagrams are
\benum
\item Vapor Pressure Diagrams
\item Temperature/composition
\item Temperature/composition (partially miscible systems).
\item Temperature/composition (liquid-solid system). 
\eenum

Most mixtures are partially miscible, meaning at certain concentrations a phase separation between the different species will occur. 

For example, if you have two different chemical species in a mixture, you may expect one of the species to freeze at a higher/lower temperature.
If this is true, as you cool the mixture one of the species should change phase before the other. 
This phase change will cause a change in concentration and can get very difficult to calculate. 

This lecture was mostly interpreting various diagrams, go through the book and read about the various plots.
Try to understand the behavior of the system in terms of the interactions that occur between the chemical species.





\end{document}