\documentclass{article}
% Packages
\usepackage[utf8]{inputenc}
\usepackage{graphicx}
\usepackage{amsmath}
\usepackage{braket}
\usepackage[margin=0.7in]{geometry}
\usepackage{hyperref}
% User-Defined Commands
\newcommand{\be}{\begin{equation}}
\newcommand{\ee}{\end{equation}}
\newcommand{\benum}{\begin{enumerate}}
\newcommand{\eenum}{\end{enumerate}}
\newcommand{\pd}{\partial}
% Title Information
\title{Chem132A: Lecture 14}
\author{Shane Flynn (swflynn@uci.edu)}
\date{11/1/17}

\begin{document}
\maketitle

\section*{Mole Fraction}
For mixtures we are usually interested in the mole fraction of the liquid phase (x$_i$), the mole  fraction of the gas phase (y$_i$), and the total mole fraction in the system  (z$_i$).
We would define the mole fraction of the system (for a component of the mixture as (n=total moles of the system).
\be
z_a = \frac{x_a + y_a}{n}
\ee

Again in this lecture we discussed various phase diagrams, look  over the  textbook  and the phase diagrams  discussed in lecture when preparing for the exam.
Be  able to identify what each axis in a plot refers too, and  the story each plot tells. 

\section*{Non-Ideal Systems}
To begin discussing non-ideal systems we started with the fugacity.
The motivation for fugacity was that a real gas system may behave like an ideal system, with some correction terms. 

\subsection*{Activity}
A more general approach to studying real chemical systems would be to  define the \textbf{Activity} (a). 
While the fugacity was a correction term defined for the gas phase, the activity is a more general definition, capturing non-ideal behavior for any any phase (including the gas phase). 

The equations we generated for ideal systems (Rault's Law for example) are very convenient (simple).
To preserve this simplicity we define the activity, such that the equations look similar (remember a * implies a pure state).  
\be
\mu_i \equiv \mu_i^* + RT \ln(a_i)
\ee
The above equation is our definition for the activity of the solvent. 
This equation is more general, and applies to any system.

Recall for an ideal system (a system obeying Raoult's Law) we wrote the partial pressure as 
\be
P_i = x_i P_i^*
\ee
And the chemical potential became 
\be
\mu_i \equiv \mu_i^* + RT \ln(x_i)
\ee

For these equations to be consistent, the activity must be defined as 
\be
a_i = \frac{P_i}{P_i^*}
\ee
In this way the solvent of a system obeying Raoult's Law simplifies to our known results for an ideal system. 

The activity plays a role similar to concentration for the real system. 
For convention (i.e. writing things in a table) we define the \textbf{Activity Coefficient} ($\gamma$), and this is what you will find  tabulated by engineers.
\be
a_i \equiv \gamma_i x_i
\ee
It should be clear that as the activity coefficient goes to 1, we reproduce our ideal system (a$_i$ $\rightarrow$ x$_i$. 
This implies an activity coefficient of 1 represents an ideal system. 
Using the activity coefficient we can re-write our chemical potential.
\be
\begin{split}
\mu_i &= \mu_i^* + RT \ln(a_i)\\
&= \mu_i^* + RT \ln(\gamma_i x_i)\\
\mu_i &= \mu_i^* + RT\ln(x_i) + RT \ln(\gamma_i)
\end{split}
\ee

\subsubsection*{Solute}
The solute can also be defined in a similar manner,  this time using Henry's Law as the Ideal metric to compare with. 
\be
\begin{split}
\mu_j &= \mu_j^* + RT\ln\left(\frac{P_j}{P_j^*}\right)\\
&= \mu_j^* + RT\ln\left(\frac{K_j}{P_j^*}\right) + RT\ln(x_j)\\
\mu_j &= \mu_j^o + RT\ln(a_j)
\end{split}
\ee
Where the last line follows form defining our reference state to be
\be
\mu_j^o \equiv \mu_j^* + RT\ln\left(\frac{K_j}{P_j^*}\right)
\ee
With these definition, an ideal solution has $K_j = P_j^*$ meaning $\mu_j^o = \mu_j^*$. 

\section*{Ionic Solutions}
In general the ideal assumption breaks down for relatively concentrated solutions. 
A major exceptions to this occurs for ionic solutions, which show deviations from ideal behavior at relatively small concentrations. 
The Coulomb Potential (electrostatics) is a long distance interaction, meaning that at even small concentrations the force generated due to electric charges effects your entire system. 

When dealing with an ionic solution, there are both cations and anions in solution. 
We cannot experimentally separate the effects of either ion, so we choose to define a mean ionic activity coefficient as 
\be
\gamma_{\pm} \equiv \sqrt{\gamma_+ \gamma_-}
\ee

The molar Gibbs Free Energy can then be written as
\be
G_m = \mu_+^{ideal} + \mu_-^{ideal} + RT\ln(\gamma_+\gamma_-)
\ee

We can then express the chemical potential for an ionic solution as
\be
\mu_i = \mu_i^{ideal} = RT\ln \gamma_\pm
\ee

\subsection*{Debye Huckel Limiting Law}
The simplest model for ionic activities (and the only one you are responsible for knowing) is the \textbf{Debye Huckel Limiting Law}.
For water at 25 $^0$C, the Law says
\be
log_{10}\gamma_\pm = -0.509|z_+z_-|\sqrt{I}
\ee
Where the coefficient (0.509) is a function of the temperature and chemical species, the z refers to the charge of the respective ions. 
The I term is known as the \textbf{Ionic Strength}
\be
I = \frac{1}{2}\sum_i z_i^2 \left(\frac{b_i}{b^o}\right)
\ee
Remember b is the molality and a superscript $^o$ refers to a reference state of your choosing. 











\end{document}