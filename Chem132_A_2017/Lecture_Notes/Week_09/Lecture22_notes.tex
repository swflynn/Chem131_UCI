\documentclass{article}
% Packages
\usepackage[utf8]{inputenc}
\usepackage{graphicx}
\usepackage{amsmath}
\usepackage{braket}
\usepackage[margin=0.7in]{geometry}
\usepackage{hyperref}
\usepackage[version=4]{mhchem}
% User-Defined Commands
\newcommand{\be}{\begin{equation}}
\newcommand{\ee}{\end{equation}}
\newcommand{\benum}{\begin{enumerate}}
\newcommand{\eenum}{\end{enumerate}}
\newcommand{\pd}{\partial}
% Title Information
\title{Chem132A: Lecture 22}
\author{Shane Flynn (swflynn@uci.edu)}
\date{11/29/17}

\begin{document}
\maketitle

\section{Catalyst}
If we compare the effects of a catalyst to the Arrhenius equation, the catalyst seems to either decrease the activation energy, or increasing the A parameter. 

An aside comment; the activation energy can also be negative.
For example, an ion interacting with a neutral atom may require a  negative activation energy.
If the molecules are going to fast there may not be time to interact through the Coulombic force.

\subsection*{Catalytic Converter}
In a combustion engine (your car) the exhaust output can be toxic.
These  outputs interact with a \textbf{Catalytic Converter}, which is a device that causes a Redox Reaction to generate less-toxic output gases. 
Un-burned hydrocarbons (CxHy), CO, and NOx are the main outputs we want to remove (the toxic gases).  
The Hydrocarbons and carbon monoxide are oxidized inside the catalytic converter.
\be
\begin{split}
\ce{C_xH_y + O2 &-> CO2 + H2O}\\
\ce{CO + \frac{1}{2}O2 &-> CO2}
\end{split}
\ee
The nitrogen-oxides are then reduced
\be
\ce{2NO + H2 -> N2 + H2O}
\ee
This process is done using a Pt/Rh catalyst on an oxide scaffold.
This is a very complicated process (hence catalytic converters are very expensive).
The air-fuel ratio effects the oxidation and reduction pathways,  and the catalyst can be poisoned easily (made useless) by lead and sulfur. 
The reason gasoline no longer has lead is because of this catalyst. 
\section{No Longer Elementary}
We want to understand the relationship between a rate law and  a reaction mechanism, when the overall reaction is not just an elementary step. 

To approach this problem consider a reaction mechanism containing two steps.
The first step is slow and the second is fast. 
\be
\begin{split}
    \textbf{Overall Reaction:}&\\
    \ce{NO2 + CO &-> NO + CO2}\\
    \textbf{Mechanism:}&\\
    \ce{NO2 + NO2 &-> NO3 + NO}  \qquad \textbf{Slow}\\
    \ce{NO3 + CO &-> NO2 + CO2} \qquad \textbf{Fast}
\end{split}
\ee
The first step is slow (and only in 1 direction) therefore it controls  the overall rate of the process; Rate = k[NO$_2$]$^2$. 

\subsection*{Reaction Intermediates}
An \textbf{Intermediate} is a chemical species that is formed and subsequently used during the reaction (it does not appear in the overall reaction). 

If we consider a different reaction and mechanism we find the following. 
\be
\begin{split}
    \textbf{Overall Reaction:}&\\
    \ce{2H2 + 2NO &-> 2H2O + N2}\\
    \textbf{Mechanism:}&\\
    \ce{2NO &<=>[k_1][k_{-1}] N2O2}  \qquad \textbf{Fast Equilibrium}\\
    \ce{H2 + N2O2 &->[k_2] H2O + N2O}  \qquad \textbf{Slow}\\
    \ce{N2O + H2 &->[k_3] N2 + H2O} \qquad \textbf{Fast}
\end{split}
\ee
The overall rate for this mechanism is still just based on the slowest step in the mechanism. 
\be
\text{Rate} = k_2[H_2][N_2O_2]
\ee
However, this rate is in terms of an intermediate which is not convenient for experimental results (we want to measure our reactants and products). 

To deal with this first, consider part 1  of the mechanism. 
Because the forward and reverse reactions are in equilibrium we can  write
\be
\begin{split}
\textbf{Rate} &= k_1[NO]^2 = k_{-1}[N_2O_2] \rightarrow \\
\textbf{Rate} &= \frac{k_1}{k_{-1}}[NO]^2 \equiv k'[NO]^2 \\
&\text{Substitute this into the slow-step rate from above}\\
\textbf{Rate} &= k_2[H_2]k'[NO]^2 \equiv k[H_2][NO]^2 \\
\end{split}
\ee

This process required knowledge about which step in the mechanism is fast or slow. 
If we are unaware of the fast/slow steps we can use the \textbf{Steady State Approximation}. 
The steady state approximation assumes the overall rate of change of an intermediate is 0.
Therefore you sum over all the rates making and using your intermediate, and set them equal to zero.
This will allow you to express intermediate concentrations in terms of the other chemical species. 
In the above mechanism we have two different intermediates, $N_2O_2$ and $N_2O$. 
The intermediate $N_2O$ appears in step 2 (where it is generated; +) and step 3 (where is is consumed; -).
Using the steady state approximation we then write
\be
\frac{d}{dt}[N_2O}] = 0 = k_2[H_2][N_2O_2] - k_3[N_2O][H_2] \rightarrow [N_2O] = \frac{k_2}{k_3}[N_2O_2]
\ee
Using the first reaction in the mechanism we can substitute in the concentration of $N_2O_2$ and find. 
\be
[N_2O] = \frac{k_1}{k_{-1}}\frac{k_2}{k_3}[NO]^2
\ee

\subsection{Mechanisms and Kinetics}
Be warned! Multiple mechanisms can give you the same rate law. 
Kinetics cannot prove a rate law, it can just show a mechanism is consistent with the rate law. 

Consider the following overall reaction
\be
\ce{2NO + O2 -> 2NO2}
\ee
One proposed mechanism for this reaction could be 
\be
\begin{split}
    \ce{NO + NO <=> N2O2}& \qquad \text{fast}\\
        \ce{N2O2 + O2 -> NO2}& \qquad \text{slow}\\
\end{split}
\ee
If we use the steady state approximation this yields
\be
\textbf{Rate} = \frac{k_2k_1}{k_{-1}}[NO]^2[O_2]
\ee

However another proposed mechanism for this reaction is
\be
\begin{split}
    \ce{NO + O2 <=> NO3}& \qquad \text{fast}\\
    \ce{NO + NO3 -> 2NO2}& \qquad \text{slow}\\
\end{split}
\ee
If we use the steady state approximation this yields
\be
\textbf{Rate} = k[NO]^2[O_2]
\ee
On the course website is an example problem and solution for an HBr mechanism that students should review. 
 
\end{document}