\documentclass{article}
% Packages
\usepackage[utf8]{inputenc}
\usepackage{graphicx}
\usepackage{amsmath}
\usepackage{braket}
\usepackage[margin=0.7in]{geometry}
\usepackage{hyperref}
\usepackage[version=4]{mhchem}
% User-Defined Commands
\newcommand{\be}{\begin{equation}}
\newcommand{\ee}{\end{equation}}
\newcommand{\benum}{\begin{enumerate}}
\newcommand{\eenum}{\end{enumerate}}
\newcommand{\pd}{\partial}
% Title Information
\title{Chem132A: Lecture 21}
\author{Shane Flynn (swflynn@uci.edu)}
\date{11/27/17}

\begin{document}
\maketitle

\section*{2 Weeks Left}
The course is nearly over, in the last two weeks we will cover all of Chapter 20, and select sections in Chapter 21.
The Final Exam is the last day of finals (Friday), 8am-10am. 

\section{Reaction Kinetics}
Most of Chapter 20 should be review material from general chemistry (if not please read the book in detail). 
We are leaving the realm of thermodynamics to begin studying kinetics. 

From intuition (classical mechanics) we expect the frequency or rate of a chemical reaction to depend on the number of times molecules/atoms collide, their orientation and speed during collision, and the concentration of species involved. 

We define the \textbf{Reaction Rate} as the number of chemical reactions occurring in the system per reaction volume per time (mol L$^{-1}$s$^{-1}$). 
\be
\ce{A -> B}
\ee
For the above chemical reaction, the Reaction Rate is the number of mol/L of A molecules converted into B every second (this is a strictly positive number). 
Physically we know our system is decreasing wrt the concentration of A therefore we define the \textbf{Rate} as
\be
\text{rate} = -\frac{d}{dt}[A]
\ee
Likewise we understand that the concentration of B increases therefore we define the rate of change of B wrt time as 
\be
\text{rate} = \frac{d}{dt}[B]
\ee

In this way we can make a general definition for the rate of a reaction as
\be
\begin{split}
\ce{aA + bB ->& eE + fF} \\
rate = -\frac{1}{a}\frac{d}{dt}[A] = -\frac{1}{b}\frac{d}{dt}[B] =& \frac{1}{e}\frac{d}{dt}[E] = \frac{1}{f}\frac{d}{dt}[F] 
\end{split}
\ee

The rate of a reaction is not a constant (and the average rate may not be a very useful statistic). 
As the reactants/products concentrations change the dynamics change (think Le Chatelier Principle).

\section{Rate Law}
Clearly, the rate must depend on concentration.
The \textbf{Rate Law} examines how the rate of a reaction depends on concentration. 
In reality the true 'rate law' for a reaction will depend on the specifics of the reaction (quantum mechanics). 
We will start to approach this topic with a simple model, the \textbf{Elementary Step}.
When we talk about an elementary step, we assume the two species simply collide with one-another. If it is a simple collision than the rate should be  linearly proportional to the concentration and we write
\be
A + B \ce{->} C, \qquad \qquad \text{Rate} = k[A][B]
\ee
Elementary reactions usually contain collisions between two things only, it is possible to get three things to collide at the same time, but more than that is really improbable. 

\section{Determining the Rate Law}
We need to be careful when discussing a general reaction, unless the reaction is an elementary step it is not possible to construct a rate law simply from stoichometry. 
Usually the rate law for a real reaction is determined  experimentally, by watching concentration  changes as a function of time. 

Consider a generic chemical reaction that IS NOT an elementary step
\be
A \ce{->} Products
\ee
This is still an extremely simple scenario (only one reactant). 
We can guess a few specific scenarios (Rate Laws).
\be
\begin{split}
\text{Rate} &= k[A]^0 \quad (\text{Zeroth Order Rate Law})\\
\text{Rate} &= k[A] \quad (\text{First Order Rate Law})\\
\text{Rate} &= k[A]^2 \quad (\text{Second Order Rate Law})
\end{split}
\ee
In general we expect the chemical equation $\ce{aA + bB -> cC + dD}$ to have a rate  of
\be
\text{Rate} = k[A]^m[B]^n
\ee
Where m and n are some constant (could be  fraction negative etc) and ARE NOT simply the stoichometric coefficients.

We need to know the mechanics of a reaction (or model a mechanism) to construct a rate law. 
Generally we break this mechanism into elementary steps and construct the rate law  from there. 

Note: all of these rate laws are written in terms  of concentration, but concentration is a function of time (it is not  a constant). 
Therefore we can talk about things like initial rates for the analysis.
Or we could make some type of time-dependent measurements and talk about averages or the rate during certain time intervals of a reaction.

\section{Integrated Rate Laws}
We can integrate the 'Order Rate Laws' from above (this is very similar to standard methods used in differential equations). 
Consider The First Order Rate Law.
\be
\begin{split}
\frac{d}{dt}[A] &= -k[A]\\
d[A] \frac{1}{[A]} &= -kdt \\
\int_{[A]_0}^{[A]} d[A] \frac{1}{[A]} &= \int_0^t-kdt \\
\ln \frac{[A]}{[A]_0} &= -kt\\
[A] &= [A]_0 e^{-kt}
\end{split}
\ee
In the same way we can solve for the integrated rate law for a second order process. 
Recall: $\int \frac{-1}{x} = \frac{1}{x^2}$. 
\be
\begin{split}
\text{Rate} &= k[A]^2 \\
-\frac{d}{dt}[A] &= k[A]^2\\
\int_0^t \frac{d[A]}{[A]^2} &= \int_0^t -kdt \\
\int_0^t \frac{1}{[A]^2}d[A] &= -k\int_0^t dt \\
\frac{1}{A}\Bigr|_0^t &= kt\Bigr|_0^t\\
\frac{1}{[A]_t} &= \frac{1}{[A]_0} + kt
\end{split}
\ee

\section{Half Life}
The \textbf{Half-Life} (t$_{1/2}$) is defined as the time required for reactant concentration to decrease by a factor of 2.
Using the first  order integrated rate law we see. 
\be
\begin{split}
\ln\left(\frac{[A]}{[A]_0}\right) &= -kt \\
\ln\left(\frac{[A]_0/2}{[A]_0}\right) &= -kt_{1/2}\\
\ln\left(\frac{1}{2}\right) &= -kt_{1/2}\\
\ln(2) &= kt_{1/2}
\end{split}
\ee
This half-life can easily be tabulated in books because it  only depends on k. 
In fact, radioactive decay is a first order process. 

For the second order reaction we complete the same process. 
\be
\begin{split}
\frac{1}{[A]_t} &= \frac{1}{[A]_0} + kt\\
\frac{2}{[A]_0} - \frac{1}{[A]_0} &=  kt_{1/2}\\
\frac{1}{[A]_0}  &= kt_{1/2}
\end{split}
\ee
This half-life is not  as useful (not tabulated in books) because it depends on the initial concentration.

As shown in the lecture, there are some other known rate-laws, and they become very mathematically complicated. 
In reality these types of things are simply determined experimentally not analytically. 

\section{Arrhenius}
If we are interested in the rate of a reaction, we know the number of collisions (which depend on the temperature and concentration) must be related to the rate.
However, not every collision is sufficient for a reaction (orientation and energy must be correct). 

If we focus in on the effect of Temperature on reaction rate (essentially all reactions are a function of temperature) we arrive at the work of Svante Arrhenius, and the \textbf{Arrhenius Equation}. 
\be
k = Ae^{\frac{-E_a}{RT}}
\ee
Here A is the 'Arrhenius constant',  and E$_a$ is the \textbf{Activation Energy}. 
From general chemistry the concept of an activation energy or an activation barrier should be familiar. 
Although thermodynamics predicts the most stable state, the activation energy may be too high and that state cannot be realized. 
The Arrhenius equation is useful, because any temperature dependent  process  can  probably be approximated by it (as a first approximation). 

As we increase the temperature of our molecules, the Boltzmann Distribution shifts to having higher energy atoms more probable.
This means at higher temperature more atoms are able to overcome the activation energy.

We can start to guess what the Arrhenius factor looks like analytically
\be
k = A e^{\frac{-E_a}{RT}} \equiv pze^{\frac{-E_a}{RT}}
\ee
In this case we can define the \textbf{Orientation Factor} (p) and the \textbf{Collision Frequency} (z).
The orientation factor  accounts for proper alignment during the collision, and the collision frequency is naturally a function of temperature.
However, this  should be roughly linear wrt temperature, the exponential term also has a temperature dependence which will 'overpower' the collision frequency temperature dependence. 
Therefore we can approximate the collision frequency as temperature independent if we are being lazy. 

\section{Controlling Reaction Rate}
Clearly the temperature can be used as a parameter for controlling reaction rate, but a different mechanism is also available. 
A \textbf{Catalyst} is a species that changes the rate  of  reaction by providing a different mechanism to complete the reaction.
A useful catalyst would provide a mechanism that has a lower activation energy than without the catalyst, therefore increasing the rate of the reaction. 
It is crucial to understand the  the catalyst does not 'change the rate' is changes the mechanism, and the new mechanism will have a different rate.
Also note, the mechanism may be different, but if the final reactants and products are the same, the thermodynamics must not change (remember state functions). 



 
\end{document}