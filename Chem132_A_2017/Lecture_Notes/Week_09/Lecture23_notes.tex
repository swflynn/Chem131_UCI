\documentclass{article}
% Packages
\usepackage[utf8]{inputenc}
\usepackage{graphicx}
\usepackage{amsmath}
\usepackage{braket}
\usepackage[margin=0.7in]{geometry}
\usepackage{hyperref}
\usepackage[version=4]{mhchem}
% User-Defined Commands
\newcommand{\be}{\begin{equation}}
\newcommand{\ee}{\end{equation}}
\newcommand{\benum}{\begin{enumerate}}
\newcommand{\eenum}{\end{enumerate}}
\newcommand{\pd}{\partial}
% Title Information
\title{Chem132A: Lecture 23}
\author{Shane Flynn (swflynn@uci.edu)}
\date{12/1/17}

\begin{document}
\maketitle

\section*{Final Exam}
The final exam will be on Friday, December 15 from 8am-10am. 

 \section*{Uni-Molecular Reactions}
 Consider some simple reaction like A $\ce{->}$ P. 
 We do not know if this is an elementary reaction.
 One proposed mechanism is the \textbf{Lindemann-Hinshelwood Mechanism}.
 \be
 \begin{split}
     A + A \ce{->} A^*  + A& \qquad \qquad \frac{
     d}{dt}[A^*] = k_a [A]^2 \\
     A + A^* \ce{->} A + A& \qquad \qquad \frac{
     d}{dt}[A^*] = -k_a' [A][A^*] \\
     A^* \ce{->} P& \qquad \qquad \frac{
     d}{dt}[A^*] = -k_b [A^*] \\
 \end{split}
 \ee
 
 Applying our steady state approximation to the intermediate A$^*$ we find
 \be
 \begin{split}
 \frac{
     d}{dt}[A^*] = 0 &= k_a [A]^2 + -k_a' [A][A^*]  + -k_b [A^*]\\
     [A^*] &= \frac{k_a[A]^2}{kb + k_a'[A]}
 \end{split}
 \ee
 We can then substitute in the intermediate concentration at steady state to our product formation rate and find. 
\be
\frac{d}{dt}[P] = k_b[A^*] = \frac{k_ak_b[A]^2}{k_b+k_a'[A]}
\ee
 If we assume the decay of A$^*$ to P is the slow step in the- mechanism than we could write $k_a'[A][A^*] > k_b[A^*]$. 
 We can use this to simplify our product rate equation. 
 \be
 \frac{d}{dt}[P] = k_b[A^*] = \frac{k_ak_b[A]^2}{k_b+k_a'[A]} \approx \frac{k_ak_b[A]}{k_a'} \equiv k_r[A]
 \ee
 In this limit (slow step assumption) we see that the reaction would look first order wrt A. 
 
 \section*{Collision Theory}
 Using \textbf{Collision Theory} we can try to predict the rate of a reaction via probability. 
 Consider a bi-molecular reaction
 \be
 A + B \ce{->} P
 \ee
 We would assume that the rate of this reaction is proportional to the number of collisions that occur between A and B molecules (assuming no complicated underlying mechanism but a simple elementary reaction). 
 \be
 \textbf{Rate} = k_r[A][B]
 \ee
 From intuition, if we assume a reaction occurs through a collision, we need  some type of cross section associated with the sizes of the atoms ($\sigma$).
 We also know the average velocities must be accounted for.
 From Maxwell-Boltzmann Statistics we know the dependence is $\braket{V} \propto \sqrt{\frac{T}{M}}$.
 If we stick in an exponential activation energy dependence and a \textbf{Steric Factor} to account for proper orientation upon collision we find
 \be
 k \approx P\sigma \sqrt{\frac{T}{M}}e^{-\frac{E_a}{RT}}
 \ee
 
 \subsubsection*{Gas Collisions}
 Recall from our previous discussions, the collision rate for a gas is given by. 
 \be
 \begin{split}
     Z_{AB} = \sigma \sqrt{\left(\frac{8k_BT}{\pi \mu}\right)}N_A^2[A][B]\\
     \sigma = \pi d^2\\
     d = \frac{1}{2}(d_A+ d_B)
 \end{split}
 \ee
 We have also defined the reduced mass $\mu$ which essentially replaced the two different masses with an average value (so we do not need to keep track of both). 
 \be
 \mu \equiv \frac{m_Am_B}{m_A+m_B}
 \ee
 
 Next class we will further elaborate this model, we will propose the cross section is a function of energy $\sigma$(E) with an intrinsic \textbf{Collision Energy} (associated with the collision) that must be overcome by the collision if a reaction is to occur. 
 
\end{document}