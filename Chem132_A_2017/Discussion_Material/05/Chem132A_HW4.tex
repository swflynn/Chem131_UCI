\documentclass{article}
\usepackage[utf8]{inputenc}

\title{Chem132A Discussion 5 Homework}
\author{Moises Romero (moiseser@uci.edu), Shane Flynn (swflynn@uci.edu) }
\date{11/4/17}

\usepackage{graphicx}
\usepackage{amsmath}
\usepackage{braket}
\usepackage[margin=0.7in]{geometry}
\usepackage[version=4]{mhchem}


\newcommand{\be}{\begin{equation}}
\newcommand{\ee}{\end{equation}}
\newcommand{\pd}{\partial}

\begin{document}

\maketitle

\section{Ionic Solutions}
In class the \textbf{Debye-Huckel Limiting Law} was introduced as a simple approximation for ionic solution activity coefficients. 
\be \label{equ1}
\log \gamma_\pm = -A|z_+z_-|\sqrt{I}
\ee
In the text A is given as an arbitrary constant. 
For aqueous solutions (water) at 25$^0$C; A = 0.509.
From this we can assume that A depends on both the chemical species (the solvent) and the temperature of the solvent. 
\be
I \equiv \frac{1}{2}\sum_i z_i^2\left(\frac{b_i}{b^0}\right)
\ee
Here the I term refers to the \textbf{Ionic Strength}. 

\subsection{Units}
In 1925 Peter Debye and Erich Huckel started their derivation for Equation \textbf{\ref{equ1}} with
\be \label{equ2}
\ln \gamma_\pm = \frac{-|z_+z_-|\kappa}{8\pi \epsilon_0 \epsilon_r k_BT}
\ee
NOTE: I am using a bit different notation from the textbook!

Because this is a logarithm (the natural log) the dimensionality must cancel (there cannot be any units).
 Show that this is true. 

\subsubsection*{Hint:}
In this expression $\kappa$ has units of inverse centimeters, $\epsilon_r$ is unit-less (relative permitivity of the solvent), and  $\epsilon_0$ is the vacuum permitivity.

\subsection{Water}
If we do some fancy math and massage equation \textbf{\ref{equ2}} we can show that it can be re-written as
\be \label{equ4}
\log \gamma_\pm = \frac{-1.8248\times 10^6 \sqrt{\rho_s}}{(D_sT)^{3/2}} |z_+z_-|\sqrt{I}
\ee
Show that this equation evaluates to the Debye-Huckel Law presented in class for water at 25$^0$C. 

\subsubsection*{Hint:}
In this questions $\rho_s$ is the density (kg/L) of the solvent, and D$_s$ is the dielectric constant of the solvent. 

\subsection{Mean Activity}
Compute the mean activity coefficient ($\gamma_\pm$) for a 0.005 (molal) aqueous solution of AlCl$_3$ (assume complete dissociation) at T = 96.85$^0$C. 


\section{Chemical Equilibrium}

\subsection{Real Gases}
Recall for a non-ideal gas, the fugacity accounts for deviations from ideal behavior. 
 
For a general reaction : 
\be
\ce{ v_AA(g) + v_BB(g) <=> v_CC(g)  v_DD(g)} 
\ee
We can describe the Gibbs Free Energy as : 
\be
\Delta_rG(T) = \Delta_rG^o(T) + RT\ln\frac{f^{(V_C)}_Cf^{(V_D)}_D}{f^{(V_A)}_Af^{(V_B)}_B}
\ee

At equilibrium $\Delta_rG = $ 0, so the expression for K$_f$(T) can be written as : 
\be
\Delta_rG^o(T) = - RTln\frac{f^{(V_C)}_Cf^{(V_D)}_D}{f^{(V_A)}_Af^{(V_B)}_B} \equiv -RT \ln K_f(T)
\ee

At low pressures we can assume the system will behave somewhat ideally, and replace partial fugacities with partial pressures to define K$_P$(T) in a similar manner. 

\bigskip

 Consider the reaction to make ammonia :
 \be
\ce{ \frac{1}{2}N_2(g) + \frac{3}{2}H_2(g) <=> NH_3 (g)}
 \ee
 The equilibrium constants of K$p$ and K$_f$ can be related by K$_\gamma$ (this is an equilibrium constant in terms of activity). 
 Develop an expression for K$_\gamma$ and evaluate it using: 
 
 \begin{center}
 \begin{tabular}{|c | c | c |} 
 \hline
 Total Pressure(Bar) & K$_p(10^{-3})$ & K$_f({10^{-3}})$  \\ [0.5ex] 
 \hline
 10 & 6.59 & 6.55  \\ 
 \hline
 30 & 6.76 & 6.59  \\
 \hline
 50 & 6.90 & 6.50  \\
 \hline
 100 & 7.25 & 6.36  \\
 \hline
 300 & 8.84 & 6.08  \\  
 \hline
 600 & 12.94 & 6.42 \\ 
 \hline
\end{tabular}
\end{center}

\subsubsection*{Hint:}
Recall the fugacity coefficient can be written as: 
\be
\gamma = \frac{f}{P}
\ee

\subsection{Equilibrium and Activity}
In general the activity is more fundamental than the fugacity. 
We can express deviations from ideal behavior for any physical state (solid, liquid, gas) as an activity coefficient. 
Gases are generally a bit simpler, and their non-ideal behavior is generally related to their pressure.
Therefore we can replace the activity of a gas with its fugacity.

In this way we can express any chemical system in terms of activity, and we can simplify the expression by substituting in expression for fugacity for any gas species. 

\bigskip

Consider the general reaction : 
\be
\ce{v_AA + v_BB <=> v_CC  v_DD}
\ee
We could write the Gibbs Free Energy of Reaction in terms of generic activity (i.e. no states specified).
\be
\Delta_rG = \Delta_rG^o(T) + RT\ln\frac{a^{(V_C)}_Ca^{(V_D)}_D}{a^{(V_A)}_Aa^{(V_B)}_B}
\ee

Let's use this model to understand a chemical system. 

\bigskip

Starting from the equation relating chemical potential to activity:
\be
\mu_i = \mu^o_i + RT\ln(a_i)
\ee

Prove the following relationship is true:
\be
\ln(a) =  \frac{\bar V}{RT}(P-1)
\ee

\subsubsection*{Hint:}
Assume a constant Temperature process and write down the fundamental equation for the Gibbs Free Energy. 

\subsection{Application:}
The change in standard molar Gibbs Free Energy for the conversion of graphite to diamond is 2.9 kJ-mol$^{-1}$ at 298K. 
The density for graphite and diamond are 2.27 g-cm$^{-3}$ and 3.52 g-cm$^{-3}$. 
At what pressure will these two compounds be at equilibrium?
(Use that equation we just had you prove to solve this question)!

\section{Enthalpy and Equilibrium}
Consider the chemical reaction
\be
\ce{2 Na(g) <=> Na2(g)}
\ee

The equilibrium constant as a function of temperature has been measured as

\begin{center}
 \begin{tabular}{|c | c |} 
 \hline
 T(Kelvin) & K   \\ 
 \hline
 900 & 1.32   \\
 \hline
 1000 & 0.47  \\
 \hline
 1100 & 0.21  \\
 \hline
 1200 & 0.10   \\  
 \hline
\end{tabular}
\end{center}

From this data estimate the standard Enthalpy of Reaction. 

\subsubsection*{Hint:}
Look up the Van't Hoff equation and try integrating it!


\end{document}