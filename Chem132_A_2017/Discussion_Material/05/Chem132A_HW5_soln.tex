\documentclass{article}
\usepackage[utf8]{inputenc}

\title{Chem132A Discussion 5 Solutions}
\author{Moises Romero, Shane Flynn}
\date{November 2017}


\usepackage{graphicx}
\usepackage{amsmath}
\usepackage{braket}
\usepackage[margin=0.7in]{geometry}
\usepackage[version=4]{mhchem}


\newcommand{\be}{\begin{equation}}
\newcommand{\ee}{\end{equation}}
\newcommand{\pd}{\partial}

\begin{document}

\maketitle

\section{Ionic Solutions; Answer Key}
In this problem we are given various equations to analyze and apply. 
DO NOT worry about knowing any of these equations other than the Debye-Huckel Limiting Law!

\subsection{Units}
\be \label{equ2}
\ln \gamma_\pm = \frac{-|z_+z_-|\kappa}{8\pi \epsilon_0 \epsilon_r k_BT}
\ee

We can re-write this equation in terms of the dimensionality of each constant. 

The unit of charge is the Coulomb (C).
The Boltzmann constant is in terms of J/K.
The vacuum permitivity is given by C$^2$/Nm$^2$. 

\be 
\ln \gamma_\pm = \frac{C^2-K-N-m^2}{m-C^2-J-K}
\ee
Note that a Joule is equal to a N-m and we do see that all the dimensions cancel. 

\subsection{Water}
If we do some fancy math and massage Equation \textbf{\ref{equ2}} we can show that it can be re-written as
\be \label{equ4}
\log \gamma_\pm = \frac{1.8248\times 10^6 \sqrt{\rho_s}}{(D_sT)^{3/2}} |z_+z_-|\sqrt{I}
\ee
Show that this equation evaluates to the Debye-Huckel Limiting Law presented in class for water at 25$^0$C. 

For water at room temperature the density $\rho_s$ is 1 kg/L and the dielectric constant for water at 25 $^0$C is 78.42.
\be \label{equ4}
\log \gamma_\pm = \frac{-1.8248\times 10^6 \sqrt{1}}{(78.42*298)^{3/2}} |z_+z_-|\sqrt{I} = -0.51 |z_+z_-|\sqrt{I}
\ee

\subsection{Mean Activity}
Compute the mean activity coefficient ($\gamma_\pm$) for a 0.005 (molal) aqueous solution of AlCl$_3$ (assume complete dissociation) at T = 96.85$^0$C. 

Let's first compute the ionic strength of the solution. 
We know each chlorine will contribute a negative 1 charge and each aluminum will contribute a positive 3 charge. 
If we take our reference state to be 1 molal, we find
\be
\begin{split}
I &\equiv \frac{1}{2}\sum_i z_i^2\left(\frac{b_i}{b^0}\right) = \frac{b}{2} \left(1\times 3^2 + 3 \times 1^2\right) \\
&= \frac{0.005}{2}(12) = 0.03 \frac{\text{mol}}{\text{kg}}
\end{split}
\ee

Next we need to compute our A coefficient. 
Note the solvent is water, but we are at a different temperature, therefore we must look up the density of water at 370k ($\rho_s$ =0.96kg/L) and the dielectric constant of water at this temperature (D$_s$ = 30.04). 
\be
A = \frac{-1.8248\times 10^6 \sqrt{0.96}}{(30.04*370)^{3/2}} = -1.526
\ee

Finally we can use the Debye-Huckel Limiting Equaton to determine the average activity for our system. 
\be
\begin{split}
\log \gamma_\pm &= -1.526 |-1\times 3|\sqrt{0.03} = -0.793\\
\end{split}
\ee

\newpage

\section{Chemical Equilibrium}
 Chemical Equilibrium for a non-ideal gas.
 
\subsection{Real Gases}
Develop an expression for K$_\gamma$ and evaluate it using data from : 
 
 \begin{center}
 \begin{tabular}{||c| c |c ||} 
 \hline
 Total Pressure(Bar) & K$_p(10^{-3})$ & K$_f(10^{-3})$  \\ [0.5ex] 
 \hline\hline
 10 & 6.59 & 6.55  \\ 
 \hline
 30 & 6.76 & 6.59  \\
 \hline
 50 & 6.90 & 6.50  \\
 \hline
 100 & 7.25 & 6.36  \\
 \hline
 300 & 8.84 & 6.08  \\  
 \hline
 600 & 12.94 & 6.42 \\ 
 \hline
\end{tabular}
\end{center}

\subsection*{Solution}
We can write the fugacity coefficient as : 
\be
f_i=\gamma_i P_i
\ee
We can then substitute this in to our express of K$_f$ to describe the relationship between the three K expressions : 
\be
K_f = \frac{{f^{(V_C)}_Cf^{(V_D)}_D}}{{f^{(V_A)}_Af^{(V_B)}_B}} = \frac{{\gamma^{(V_C)}_PP^{(V_C)}_C\gamma^{(V_D)}_DP^{(V_D)}_D}}{{\gamma^{(V_A)}_AP^{(V_A)}_A\gamma^{(V_B)}_B}P^{(V_B)}_B} = \left(\frac{\gamma^{(V_C)}_C\gamma^{(V_D)}_D}{\gamma^{(V_A)}_A\gamma^{(V_B)}_B}\right)
\left(\frac{P^{(V_C)}_CP^{(V_D)}_D}{P^{(V_A)}_AP^{(V_B)}_B}\right) = K_\gamma \dot K_P
\ee

So we can calculate K$_\gamma$ at each given pressure to get : 

\begin{center}
 \begin{tabular}{||c | c  ||} 
 \hline
 Total Pressure(Bar) & K$_\gamma$  \\ [0.5ex] 
 \hline\hline
 10 & .994   \\ 
 \hline
 30 & .975   \\
 \hline
 50 & .942  \\
 \hline
 100 & .877  \\
 \hline
 300 & .688  \\  
 \hline
 600 & .4996 \\ 
 \hline
\end{tabular}
\end{center}

As expected, the higher the pressure, the less ideal the system becomes. 

\subsection{Equilibrium expressed in terms of activity}
The change in standard molar Gibbs Free Energy for the conversion of graphite to diamond is 2.9 kJ-mol$^{-1}$ at 298K. The density for graphite and diamond are 2.27 g-cm$^{-3}$ and 3.52g-cm$^{-3}$. At What pressure will these two compounds be at equilibrium? 

\subsection*{Solution}
First we will start by finding an expression for the activity in terms of pressure. 
Them we will solve for the pressure.

To prove the relationship we will start with the fundamental equation for Gibbs Free Energy. 
At constant temperature the fundamental equation can be expressed as follows : 
\be
dG = VdP - SdT \rightarrow dG = d\mu = VdP
\ee
Combining this result with: 
\be
d\mu = RTdln(a) 
\ee
We see that 
\be
\bar V dP = RTdln(a) 
\ee

We then need to simply integrate both sides to find our relationship.
\be
d\ln(a) = \frac{\bar V}{RT}dP => \int^{a}_{a=1} d\ln(a) = \frac{1}{RT} \int^{P}_{1} \bar V P 
\ee
We integrate from an assigned standard state (a=1 and P=1 bar) , once we integrate we are left with : 
\be
\ln(a) =  \frac{\bar V}{RT}(P-1)
\ee

We can now use this relationship to determine our pressure at equilibrium. 
\be
\Delta_rG^o = -RT\ln(K_a) = -RT\ln\left( \frac{a_{diamond}}{a_{graphite}}\right) = -RT\left[\frac{\Delta \bar V}{RT}(P-1)\right]
\ee
We can rearrange this equation : 
\be
\frac{2900 Jmol^{-1}}{(8.3145 Jmol^{-1}K^{-1})(298 K)} = - \frac{(3.41cm^3mol^{-1} - 5.29 cm^3mol^{-1}(\frac{1dm^3}{1000cm^3})(P-1)bar}{(.083145dm^3barmol^{-1}K^{-1}(298K)}
\ee 
Which can then be simplified : 
\be
1.17 = - (-7.59 x 10^{-5})(P-1) 
\ee
\be
P \approx 15,000 \quad Bar 
\ee

\section{Enthalpy and Equilibrium}
Consider the chemical reaction
\be
\ce{2 Na(g) <=> Na2(g)}
\ee

The equilibrium constant as a function of temperature has been measured as

\begin{center}
 \begin{tabular}{|c | c |} 
 \hline
 T(Kelvin) & K   \\ 
 \hline
 900 & 1.32   \\
 \hline
 1000 & 0.47  \\
 \hline
 1100 & 0.21  \\
 \hline
 1200 & 0.10   \\  
 \hline
\end{tabular}
\end{center}

From this data estimate the standard Enthalpy of Reaction. 

\subsection*{Solution}
We can derive the Van't Hoff equation by starting with the Gibbs-Helmholtz equation.
\be
\left(\frac{\pd (\Delta_rG^o/T)}{\pd T}\right) = \frac{-\Delta_rH^o}{T^2}
\ee
At equilibrium we know that 
\be
\Delta_rG^o = -RT\ln K
\ee
Equating these expressions we find
\be
\left(\frac{\pd \ln K}{\pd T}\right) = \frac{\Delta_rH^o}{T^2 R}
\ee
Let's now integrate the expression
\be
\begin{split}
\int_{\ln K_1}^{\ln K_2}\left(\frac{\pd \ln K}{\pd T}\right) &= \int_{T_1}^{T_2}\frac{\Delta_rH^o}{RT^2}\\
\ln K_2 - \ln K_1 &= -\frac{\Delta_r H^o}{R}\left(\frac{1}{T_2}- \frac{1}{T_1} \right)
\end{split}
\ee
This equation means we can plot ln K versus 1/T, the slope will be $\frac{\Delta_rH^o}{R}$

\end{document}