\documentclass{article}
\usepackage[utf8]{inputenc}

\title{Chem132A Discussion 1 Solutions}
\author{swflynn }
\date{September 2017}


\usepackage{graphicx}
\usepackage{amsmath}
\usepackage{braket}
\usepackage[margin=0.7in]{geometry}


\newcommand{\be}{\begin{equation}}
\newcommand{\ee}{\end{equation}}
\newcommand{\pd}{\partial}

\begin{document}

\maketitle


\section{Ideal Gas vs Real Gas}

\subsection{Derivatives}
For the following function solve for $\frac{\pd f(x,y,z)}{\pd x}$, $\frac{\pd f(x,y,z)}{\pd y}$, and $\frac{\pd f(x,y,z)}{\pd z}$
\be
f(x,y,z)=x^2y + y^2x
\ee
\subsection*{Solution:}
For $\frac{\pd f(x,y,z)}{\pd x}$ we want to see how the function changes with respect to (wrt) x. Therefore other variables are considered constants and we take the derivative wrt x. 
\be
\frac{\pd f(x,y,z)}{\pd x} f(x,y,z)= \frac{\pd}{\pd x}\left(x^2y + y^2x\right) = 2xy + y^2 
\ee
For $\frac{\pd f(x,y,z)}{\pd y}$ we want to see how the function changes with respect to y. This time we take the derivative of the y terms as the x terms remain constant.
\be
\frac{\pd f(x,y,z)}{\pd y} f(x,y,z)=\frac{\pd}{\pd y}\left(x^2y + y^2x\right) = x^2 + 2yx
\ee
For $\frac{\pd f(x,y,z)}{\pd z}$ we want to see how the z terms affect the function, so the x and y terms behave as constants. 
\be
\frac{\pd f(x,y,z)}{\pd z} f(x,y,z)=\frac{\pd}{\pd z}\left(x^2y + y^2x\right) = 0
\ee
The value here is 0.
There is no z term within the function and the derivative of a constant is 0. Meaning we should really write f(x,y) as there is no z dependence.

\subsection{Ideal Gas Law}
For 1 mol of Argon gas at T=300 K and V=10 L, calculate $\frac{\pd p}{\pd T}$ and $\frac{\pd p}{\pd V}$ using the ideal gas law.
\be
pV=nRT
\ee
What does each solution mean? 

\subsection*{Solution:}
Because we only have 1 mole of gas (n=1), we can rearrange our equation: 
\be
p=\frac{RT}{V}
\ee
For $\frac{\pd p}{\pd T}$ : 
\be
\frac{\pd p}{\pd T} = \frac{R}{V}
\ee
With the knowledge of the derivation we can now plug in values.
\be
\frac{\pd}{\pd T} = \frac{R}{V} = \frac{8.314}{10} = .8124 (\text{kPa/K})
\ee
This means at conditions of 300 K and 10 L, every 1K increase in temperature increases the pressure by .8124 kPa. \\

For $\frac{\pd p}{\pd V}$
\be
\frac{\pd p}{\pd V} = \frac{-RT}{V^2}
\ee
Once again we plug in the values given : 
\be
\frac{\pd p}{\pd V} = \frac{-RT}{V^2} = \frac{(-8.314)(300)}{10^2} = -24 (\text{kPa /L})
\ee
For the conditions of T=300 K and V=10 L; every increase of 1 L causes the pressure to decreases by 24 kPA. 

\subsection{Real Gas Law vs Ideal Gas Law}
A. Calculate the pressure exerted by N$_2$ at 300K for the molar volumes of 0.100 L-mol$^{-1}$ and 250 L-mol$^{-1}$. 
Use the  ideal gas law and van der Walls equation. 
The parameters for N$_2$ are a=1.370 bar-dm$^6$-mol$^{-2}$ and b=0.0387 dm$^3$-mol$^{-1}$.
\be
p=\frac{RT}{V_m - b}-\frac{a}{V^2_m}
\ee

\subsection*{Solution:}
This problem is meant to show the differences between the two gas laws.
Remember the molar volume is 
\be
V_m = \frac{V}{n}
\ee

Ideal Gas Law calculations for 

V$_m$ = 0.100 L-mol$^{-1}$ 

\be
p = \frac{RT}{V_m} = \frac{(.08314 \text{L-bar-mol}^{-1}\text{K}^{-1})(300 \text{K})}{0.100 \text{L/mol}} = 249 \text{bar}
\ee

V$_m$ = 250 L-mol$^{-1}$ 

\be
p = \frac{RT}{V_m} = \frac{(.08314\text{L-bar- mol}^{-1}\text{K}^{-1})(300\text{K})}{250\text{L/mol}} = 9.98 \cdot 10^{-2} \text{bar}
\ee

Real Gas Law calculations for 

V$_m$ = 0.100 L-mol$^{-1}$ 
\be
p=\frac{RT}{V_m - b}-\frac{a}{V^2_m} = \frac{(.08314\text{L-bar-mol}^{-1}\text{K}^{-1})(300\text{K})}{0.100\text{L-mol}^{-1} - 0.0387\text{dm}^3\text{mol}^{-1}} - \frac{1.370\text{bar-dm}^6\text{mol}^{-2}}{(0.1\text{L-mol}^{-1})^2} = 270 \text{bar}
\ee

V$_m$ = 250 L-mol$^{-1}$ 
\be
p=\frac{RT}{V_m - b}-\frac{a}{V^2_m} = \frac{(.08314\text{L-bar-mol}^{-1}\text{K}^{-1})(300\text{K})}{250\text{L-mol}^{-1} - 0.0387\text{dm}^3\text{mol}^{-1}} - \frac{1.370\text{bar-dm}^6\text{mol}^{-2}}{(250\text{L-mol}^{-1})^2} = 9.98 \cdot 10^{-2} \text{bar}
\ee
Recall that value a corrects for particle interactions and value b corrects for volume. 

Understanding the significance of these values : 

If $p_{\text{real}} > p_{\text{ideal}}$: this suggests that the atoms are repelling one another, generating a larger outward force. 
With a smaller volume the atoms are much closer to each-other, which will force them to interact more.

If $p_{\text{real}}<p_{\text{ideal}}$:
this suggests the atoms are attracted to one-another, decreasing the overall outward force.
At larger distances dispersion forces will dominate and the atoms will exhibit a slight attractive force.

If p$_{\text{real}}$ = p$_{\text{ideal}}$ then the real gas is behaving ideally.
This suggests the particles are not interacting much.
We observe this at a molar volume of 250L-mol$^{-1}$. 

At low pressure the atoms simply do not see each-other very often, so their interactions become independent. 
 
\section{Meeting The Maxwell-Boltzmann Distribution}

\subsection{Integration By Parts (IBP)}
Consider the function
\be
y(x) = a(x) b(x)
\ee
Starting with the right hand side (RHS) of this equation, derive the formula for integration by parts.

\subsection*{Solution:}
Integration by parts comes from the product rule for derivatives as we will show below. 

Let's start by taking the derivative of y(x).
\begin{equation}
    \frac{d}{dx}\left[a(x)b(x)\right] = a(x)b'(x) + b(x)a'(x)
\end{equation}
Now take the anti-derivative of both sides. 
\begin{equation}
    \begin{split}
        \int\frac{d}{dx}\left[a(x)b(x)\right]dx &= \int a(x)b'(x) dx+ \int b(x)a'(x) dx\\
         \left[a(x)b(x)\right] &= \int a(x)b'(x) dx+ \int b(x)a'(x)dx 
    \end{split}
\end{equation}
The integral and derivative operations are inverses from the fundamental theorem of calculus, allowing us to cancel terms. 
Now we can simply rearrange our equation to find a formula for our integral of interest. 
\begin{equation}
    \int a(x)b'(x) dx =  \left[a(x)b(x)\right] - \int b(x)a'(x) dx
\end{equation}
This is our integration by parts formula, a direct result of the product rule of derivatives. 
It is common for IBP to be written as 
\be
\int udv = uv - \int vdu
\ee

\subsection{An Interesting Integral}
Compute the following integral.  
\be
y(x) = \int_0^\infty x^{n}e^{-x}dx 
\ee
\textbf{Hint:} Use IBP twice, and then stop. 

\subsection*{Solution}
To solve this integral we notice that the exponential will not change in power, but we can reduce the power of the x term. 
This general format suggests to do integration by parts (IBP), so let: 
\begin{center}
  \begin{tabular}{ | l | c | }
    \hline
    u = x$^n$ & v = -e$^{-x}$ \\ \hline
    du = nx$^{n-1}$ dx & dv = e$^{-x}$dx  \\
    \hline
  \end{tabular}
\end{center}
Using our IBP substitutions we can evaluate this integral (note the uv term goes to 0 when you evaluate both limits).
\begin{equation}
\begin{split}
    \int_0^\infty x^{n}e^{-x}dx &= x^n(-e^{-x}) \Big|_0^\infty - \int_0^\infty -e^{-x}nx^{n-1}dx \\
    &= n\int_0^\infty x^{n-1}e^{-x}dx
    \end{split}
\end{equation}
And here we start to see something interesting, let's repeat the IBP one more time to better see the pattern.
\begin{center}
  \begin{tabular}{ | l | c | }
    \hline
    u = x$^{n-1}$ & v = -e$^{-x}$ \\ \hline
    du = n-1x$^{n-2}$ dx & dv = e$^{-x}$dx  \\
    \hline
  \end{tabular}
\end{center}
\begin{equation}
\begin{split}
n\int_0^\infty x^{n-1}e^{-x}dx &= n\left[ x^{n-1}(-e^{-x})\right]\Big|_0^\infty - n\int_0^\infty (n-1)x^{n-2}e^{-x}dx \\
n\int_0^\infty x^{n-1}e^{-x}dx & \\ &=n(n-1)\int_0^\infty x^{n-2}e^{-x}dx \\
\end{split}
\end{equation}
We now see the pattern, if we keep repeating this process we will keep integrating out n-i terms n(n-1)(n-2)(n-3)...1
Which looks just like the factorial function!
(When you reach the limit (n=n) the last integral will evaluate to 1). 
\be
\int_0^\infty e^{-x}dx = 0 - (-1) = 1
\ee

\subsection{Finally; The Distribution}
Let's now look at the Maxwell-Boltzmann Speed Distribution, f($\nu$).
\be
f(\nu) = 4\pi \left(\frac{M}{2\pi RT}\right)^{\frac{3}{2}} \nu^2 e^{\frac{-M\nu^2}{2RT}}
\ee
Please show that this distribution is normalized. 

\textbf{Hint:} To do this, take the following integral as true (meaning you do not need to prove it, just use it). 
\be
\int_0^\infty x^2 e^{-ax^2} dx = \frac{1}{4a}\left(\frac{\pi}{a}\right)^{\frac{1}{2}}
\ee

\subsection*{Solution}
We need to integrate over all of space, and show this integral is equal to 1. 
The equation given has already changed the bounds of the integral due to the even nature of the function, so take the integral from 0 to infinity instead. 
\be
\int_{-\infty}^\infty f(\nu)d\nu = \int_0^\infty 4\pi \left(\frac{M}{2\pi RT}\right)^{\frac{3}{2}} \nu^2 e^{\frac{-M\nu^2}{2RT}}d\nu = 4\pi \left(\frac{M}{2\pi RT}\right)^{\frac{3}{2}} \int_0^\infty \nu^2 e^{\frac{-M\nu^2}{2RT}}d\nu
\ee
Pulling out the constants, the integral we must compute is of the same form as the hint. 
Let a = $\frac{m}{2RT}$, using this substitution we can then use the integral provided in the hint to solve the problem.
\be
= 4\pi \left(\frac{M}{2\pi RT}\right)^{\frac{3}{2}} \frac{1}{4a}\left(\frac{\pi}{a}\right)^{\frac{1}{2}} = 4\pi \left(\frac{M}{2\pi RT}\right)^{\frac{3}{2}}\left(\frac{2RT}{4M}\right)\left(\frac{2\pi RT}{m}\right)^{\frac{1}{2}}
\ee
Finally we can use basic exponent manipulations $x^{3/2} = x x^{1/2}$, to expand the first term and do the algebra. 
\be
= 4\pi \left(\frac{M}{2\pi RT}\right) \left(\frac{M}{2\pi RT}\right)^{\frac{1}{2}}\left(\frac{2RT}{4M}\right)\left(\frac{2\pi RT}{m}\right)^{\frac{1}{2}} = 1
\ee

\subsection{Kinetic Energy}
The Maxwell-Boltzmann Distribution can be rewritten in terms of kinetic energy (E$_k$) not speed. 
\be
f(E_k) = \frac{2\pi}{(\pi RT)^{3/2}} \left(E_k\right)^{\frac{1}{2}} e^{\left(-E_k/RT\right)} 
\ee
Again show this distribution is normalized.
\textbf{Hint:} The following integral will make your life easier. 
\be
\int_0^\infty x^{\frac{1}{2}} e^{-ax} dx = \frac{1}{2a}\left(\frac{\pi}{a}\right)^{\frac{1}{2}}
\ee

\subsection*{Solution}
This problem is extremely similar to the previous one. 
We start by writing down our integral and pulling out constants. 
\be
\int_0^\infty \frac{2\pi}{(\pi RT)^{3/2}} \left(E_k\right)^{\frac{1}{2}} e^{\left(-E_k/RT\right)} dE_k = \frac{2\pi}{(\pi RT)^{3/2}} \int_0^\infty \left(E_k\right)^{\frac{1}{2}} e^{\left(-E_k/RT\right)} dE_k
\ee
Letting a = $\frac{1}{RT}$ we can directly substitute in our integral expression provided in the hint and we find. 
\be
\int_0^\infty f(E_k)dE_k = \frac{2\pi}{(\pi RT)^{3/2}} \left(\frac{\pi RT}{2}\right) \left(\pi RT\right)^{\frac{1}{2}} = 1
\ee

\section{Our First Program}

\subsubsection*{}
Compute the expected value (mean), variance, and standard deviation for a population of water molecules at 100$^0$C,
by integrating over the distribution for all speeds (0 to $\infty$). \\

\subsection{The Expected Value} 
\be
E[\nu] = \int_{\nu_1}^{\nu_2} \nu f(\nu) d\nu 
\ee
\textbf{Comment:} The expected value (average value) is just a number!

\subsection{The Variance}
\be
Var[\nu] = \int_{\nu_1}^{\nu_2} \left(\nu-E[\nu]\right)^2 f(\nu) d\nu 
\ee

\subsection{The Standard Deviation}
\be
\sigma = \sqrt{Var[\nu]}
\ee

\subsection*{Solution}
Please see the github for a solution written in Python and a separate solution written in Mathematica. 
Please download the file and run it on your local computer to see our solutions. 
If you are unable or do not know how to do this, please come ask!

\end{document}
