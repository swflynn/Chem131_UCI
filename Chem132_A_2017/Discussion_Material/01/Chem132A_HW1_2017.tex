\documentclass{article}
\usepackage[utf8]{inputenc}

\title{Chem132A Discussion 1 Homework}
\author{Moises Romero (moiseser@uci.edu), Shane Flynn (swflynn@uci.edu) }
\date{10/1/17}


\usepackage{graphicx}
\usepackage{amsmath}
\usepackage{braket}
\usepackage[margin=0.7in]{geometry}


\newcommand{\be}{\begin{equation}}
\newcommand{\ee}{\end{equation}}
\newcommand{\pd}{\partial}

\begin{document}

\maketitle

\section*{Chem132A Discussions and Exams}
Greetings Chem132A, and welcome to your first discussion section 'homework'. 
These assignments will not be graded, however, they will be more representative of exam questions than Webassign.
It is possible that some exam questions will be variations of problems taken from the discussion sections. 
We highly encourage you to take a look at the problems before discussion, and come to discussion with comments/questions. 

The discussion question may be somewhat mathematically intensive for some of you.
The language of mathematics is necessary for understanding physical systems, and will be essential for understanding thermodynamics.
Use these homework assignments to talk with students and instructors to fill any gaps in your mathematics. 


\section{Ideal Gas vs Real Gas}
In chapter 1 the \textbf{Ideal Gas Law} was introduced.
The Ideal Gas Law assumes that the atoms within the system do not interact (low pressure large volume), but in a real system this is never the case.
Any equation describing a real gas will need to account for attractions and repulsion's between atoms.

\subsection{Derivatives}
For the following function solve for $\frac{\pd f(x,y,z)}{\pd x}$, $\frac{\pd f(x,y,z)}{\pd y}$, and $\frac{\pd f(x,y,z)}{\pd z}$
\be
f(x,y,z)=x^2y + y^2x
\ee

\subsection{Ideal Gas Law}
For 1 mol of Argon gas at T=300K and V=10L calculate $\frac{\pd p}{\pd T}$ and $\frac{\pd p}{\pd V}$  using the ideal gas law:
\be
pV=nRT
\ee
What does each solution tell us (think about the physical interpretation of a derivative)?

\subsection{Real Gas Law vs Ideal Gas Law}
 Calculate the pressure exerted by N$_2$ at 300K for the molar volumes of .100 L-mol$^{-1}$ and 250 L-mol$^{-1}$. 
Use the van der Wall's gas law and the ideal gas law. 
The parameters for N$_2$ are a =  1.370 bar-dm$^6$-mol$^{-2}$ and b =.0387 dm$^3$-mol$^{-1}$
\be
p=\frac{RT}{V_m - b}-\frac{a}{V^2_m}
\ee

\subsubsection*{}
What do parameters a and b account for?
Explain why (physically) we get different results using these two equations? 
 
 \subsubsection*{}
 What does it mean if:
 \begin{enumerate}
 \item $p_{\text{real}}>p_{\text{ideal}}$
 \item $p_{\text{real}}<p_{\text{ideal}}$
 \item  $p_{\text{real}}=p_{\text{ideal}}$
 \end{enumerate}
 
\section{Meeting The Maxwell-Boltzmann Distribution}
In chapter 1 the \textbf{Kinetic-Molecular Theory} was introduced, and a very important distribution, The \textbf{Maxwell-Boltzmann Distribution} of speeds was presented. 
The purpose of this question is to understand the expression, and to become familiar with some mathematical machinery useful in thermodynamics (and Quantum Mechanics, and Statistical Mechanics). 

\subsection{Integration By Parts (IBP)}
Most physical properties (the things you measure in a lab) can be determined by solving an integral. 
Integrating complicated expressions is an art, and here we will explore one such method.

Consider the function 
\be
y(x) = a(x) b(x)
\ee
Starting with the right hand side (RHS) of this equation, derive the formula for integration by parts.
Your answer should be in terms of a(x), b(x) and their associated integrals and/or derivatives. 

\subsection{An Interesting Integral}
Now that we understand how IBP works, let's use it to compute an interesting integral. 
\be
y(x) = \int_0^\infty x^{n}e^{-x}dx 
\ee
\textbf{Comment:} Use IBP twice, and then \textbf{STOP}. \\
Can you see a pattern?
What function does this look like (if you are confused try doing another IBP)?

\subsection{Finally; The Distribution}
Let's now look at the Maxwell-Boltzmann Speed Distribution, f($\nu$).
\be
f(\nu) = 4\pi \left(\frac{M}{2\pi RT}\right)^{\frac{3}{2}} \nu^2 e^{\frac{-M\nu^2}{2RT}}
\ee
The text claims that this distribution can be used to determine the probability of the molecules having a speed in the range $\nu$ to $\nu + d\nu$. 
To interpret any distribution as a probability, the integral over all of space must be equal to 1.
Please show that this distribution is normalized. 

\textbf{Hint:} To do this, take the following integral as true (meaning you do not need to prove it, just use it). 
\be
\int_0^\infty x^2 e^{-ax^2} dx = \frac{1}{4a}\left(\frac{\pi}{a}\right)^{\frac{1}{2}}
\ee

\subsection{Kinetic Energy}
The Maxwell-Boltzmann Distribution can be rewritten in terms of kinetic energy (E$_k$). 
\be
f(E_k) = \frac{2\pi}{(\pi RT)^{3/2}} \left(E_k\right)^{\frac{1}{2}} e^{\left(-E_k/RT\right)}
\ee
Again show this distribution is normalized.\\
\textbf{Hint:} The following integral will make your life easier. 
\be
\int_0^\infty x^{\frac{1}{2}} e^{-ax} dx = \frac{1}{2a}\left(\frac{\pi}{a}\right)^{\frac{1}{2}}
\ee

\section{Our First Program}
For this course (and in real life) you  need to take advantage of computers.
Each week, we will provide a simple problem that should be done using some form of mathematics software.
Exams will be done in class and \textbf{ Will NOT} require any form of programming.
However, the concepts we cover in these questions are still fair game for the exam (so at-least compile and run the solution we provide). 

For this course we suggest you utilize \textbf{Mathematica} (Moises) or \textbf{Python} (Shane) to complete these questions.
We will provide two solutions, one in each language, and you are welcome to look at either (you do not need to use both).
You are welcome to use any other programming language you like (C++, Fortran, Matlab, Ruby, etc), but solutions will only be provided for Python and Mathematica. 

\subsubsection*{}
\textbf{DO NOT} use a spreadsheet (excel), or pen and paper for these calculations. 

\subsubsection*{}
In question 2 we showed that the Maxwell-Distribution is normalized and can be used as a probability distribution.
Please compute the expected value, variance, and standard deviation for a population of water molecules at 100$^0$C,
by integrating over the distribution for all speeds (0 to $\infty$). \\
\textbf{Comment:} There are other ways of computing these properties, we are asking you to compute them numerically with the following integrals.

\subsection{The Expected Value} 
\be
E[\nu] = \int_{\nu_1}^{\nu_2} \nu f(\nu) d\nu 
\ee
\textbf{Comment:} The expected value (average value) is just a number!

\subsection{The Variance}
\be
Var[\nu] = \int_{\nu_1}^{\nu_2} \left(\nu-E[\nu]\right)^2 f(\nu) d\nu 
\ee

\subsection{The Standard Deviation}
\be
\sigma = \sqrt{Var[\nu]}
\ee

\end{document}