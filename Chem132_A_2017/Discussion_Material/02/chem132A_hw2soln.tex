\documentclass{article}
\usepackage[utf8]{inputenc}

\title{Chem132A Discussion 2 Solutions}
\author{Moises Romero, Shane Flynn}
\date{October 2017}


\usepackage{graphicx}
\usepackage{amsmath}
\usepackage{braket}
\usepackage[margin=0.7in]{geometry}


\newcommand{\be}{\begin{equation}}
\newcommand{\ee}{\end{equation}}
\newcommand{\pd}{\partial}

\begin{document}

\maketitle

\section{Path-Dependent and State Functions}
Calculate the change in Internal Energy ($\Delta$U), heat (q), and work (w) associated with traveling from point 1 to point 3.
Assume an ideal gas for all of the calculations, and that each pathway is done reversibly. 

\subsubsection*{Comment:}
To prove that the Internal Energy U(T,V) can actually be written as U(T) is a non-trivial task requiring Statistical Mechanics. 
We can somewhat justify the statement using the equipartition theorem. 
The equipartition theorem essentially claims that every degree of freedom in your system contributes a factor of $\frac{1}{2}$kT to the total energy of your system. 
This equation only depends on T as a variable and not V.

\subsection{Pathway 1 $\Rightarrow$ 3}
Starting at Point 1, compute the heat, work, and internal energy associated with traveling pathway A to arrive at point 3.

\subsubsection{Solution}
For this transition we are moving from (P$_1$,V$_1$,T$_1$) to (P$_2$,V$_2$,T$_1$).
Applying the hint we see that there is no change in temperature, therefore
\be
\Delta U = q + w \Rightarrow 0 = q + w \Rightarrow w = -q
\ee
We now need to simply calculate either the heat or work for an ideal gas that is changing in both pressure and volume. 
Because the process is done reversibly, we can easily determine the work.
\be
w \equiv \int -P_{ext} dV = \int -P_{gas}dV = -nRT_1\ln\frac{V_2}{V_1}
\ee
As stated above this implies that
\be
q = nRT_1\ln\frac{V_2}{V_1}
\ee

\subsection{Pathway 1 $\Rightarrow$ 2 $\Rightarrow$ 3}
Starting at Point 1, compute the heat, work, and internal energy associated with traveling pathway D and E to arrive at point 3.
 
 \subsubsection{Solution}
 This pathway goes through the following operations, (P$_1$,V$_1$,T$_1$) to (P$_1$,V$_2$,T$_3$) to (P$_2$,V$_2$,T$_1$). 
 For the first transition (Pathway D), there is no change in the pressure, therefore pressure must be constant. 
 With a constant pressure we can evaluate the work for the process as 
 \be
 w_D = -P_{ext}(V_2 - V_1)
 \ee
 From the First Law we can then write the expression for heat as 
 \be
 \begin{split}
 q &= U - w \\
 q_D &= \int_{T_1}^{T_3} C_V(T)dT + P_{ext}(V_2-V_1)
 \end{split}
 \ee
 
 Next we evaluate pathway E which occurs at constant volume, therefore no work is done. 
 \be
 q_E = \Delta U_E = \int_{T_3}^{T_1} C_v(T) dT
 \ee
 
 We can now combine all of the calculations.
\be
q_{123} = q_D + q_E = \int_{T_1}^{T_3} C_V(T)dT + P_{ext}(V_2-V_1) + \int_{T_3}^{T_1} C_v(T) dT = P_{ext}(V_2-V_1)
\ee

\be
w_{123} = w_D + w_E = -P_{ext}(V_2 - V_1) + 0 = -P_{ext}(V_2 - V_1)
\ee

\be
\Delta U_{123} = q_{123} + w_{123} = 0
\ee
 
 
 \subsection{Pathway 1 $\Rightarrow$ 4 $\Rightarrow$ 3}
Starting at Point 1, compute the heat, work, and internal energy associated with traveling pathway B and C to arrive at point 3.
Take pathway B to be a reversible, adiabatic expansion. 

\subsubsection{Solution}
For this process we are moving from (P$_1$,V$_1$,T$_1$) to (P$_3$,V$_2$,T$_2$) to (P$_2$,V$_2$,T$_1$). 
We are told that pathway B is done adiabatically, therefore no heat transfer is involved.
\be
d U_B = \delta w_B
\ee

Pathway C considers a constant volume process. 
If we consider only PV work this means
\be
d U_C = \delta q_C
\ee

Recall the definition of constant volume heat capacity as (not to be confused with pathway C)
\be
\begin{split}
C_V (T) &= \left(\frac{\pd U}{\pd T}\right)_V \Rightarrow\\
dU &= C_V(T) dT \Rightarrow \\
\Delta U_C &= \int_{T_2}^{T_1} C_V(T) dT
\end{split}
\ee

Returning to pathway B we can determine the Internal Energy in the same way.
This confuses many students, although pathway B is not at constant Volume, the Internal Energy is a state function, so the statement must still be true.
\be
w_B = \Delta U_B = \int_{T_1}^{T_2} C_v(T) dT
\ee

We can now combine all of the calculations.
\be
q_{143} = q_B + q_C = 0 + \int_{T_2}^{T_1} C_V(T) dT = \int_{T_2}^{T_1} C_V(T) dT
\ee

\be
w_{143} = w_B + w_C = \int_{T_1}^{T_2} C_v(T) dT + 0 = \int_{T_1}^{T_2} C_v(T) dT
\ee

\be
\Delta U_{143} = \Delta U_B + \Delta U_C = \int_{T_1}^{T_2} C_v(T) dT + \int_{T_2}^{T_1} C_V(T) dT = 0
\ee

So we see that the path functions are not 0, however, the Internal Energy is. 
As stated above, U(T,V) $\Rightarrow$ U(T), therefore its change must be 0 for this path, because there is no change in temperature.

\section{Equations of State and Derivatives}

\subsection{A Unique EOS}
Consider the arbitrary EOS
\be
\frac{Pv}{RT} = 1 + xP + yP^2
\ee
Use this EOS, compute the isothermal compressibility ($\kappa_T$). 
\be
\kappa_T \equiv -\frac{1}{V}\left(\frac{\pd V}{\pd P}\right)_T
\ee

\subsubsection{Solution}
First let's rearrange our EOS to compute the derivative.
\be
\frac{Pv}{RT} = 1 + xP + yP^2 \Rightarrow V = RT\left(\frac{1}{p} + x + yP\right)
\ee
Next we can compute the derivative (this should be very easy...)
\be
\begin{split}
\left(\frac{\pd V}{\pd P}\right)_T &= RT \left(\frac{\pd }{\pd P}\right)_T \left(\frac{1}{p} + x + yP\right) \\
&= \frac{-RT}{P^2} + yRT
\end{split}
\ee
With the derivative we can now compute the compressibility:
\be
\kappa_T = \frac{RT}{P^2V} - \frac{yRT}{V}
\ee


\subsection{A Familiar EOS}
Using the van der Walls EOS for a gas, compute the isothermal compressibility. 

\subsubsection{Solution}
Start by using the hint, this will save us time.
\be
\kappa_T = -\frac{1}{V}\left(\frac{\pd V}{\pd P}\right)_T = -\frac{1}{V}\left(\frac{1}{\frac{\pd P}{\pd V}}\right)_T 
\ee
Now we can compute the necessary derivative. 
\be
\left(\frac{\pd P}{\pd V}\right)_T = \frac{\pd}{\pd V}\left(\frac{nRT}{(V-nB)} - a\left(\frac{n}{V}\right)^2\right)_T = \frac{-nRT}{(V-nb)^2} + \frac{2an^2}{V^3}
\ee
With the derivative we can now compute the compressibility. 
\be
\kappa_T = \left(\frac{-1}{V}\right) \frac{1}{ \left[\frac{-nRT}{(V-nb)^2} + \frac{2an^2}{V^3}\right]} = \left(\frac{-1}{V}\right)\left[\frac{V^3(V-nb)^2}{nrTV^3 - 2an^2(V-nb)^2}\right] = \frac{V^2(V-nb)^2}{nRTV^3 - 2an^2(V-nb)^2}
\ee

\subsection{Internal Pressure}
\be
\pi_T \equiv \left(\frac{\pd U}{\pd V}\right)_V = T\left(\frac{\pd P}{\pd T}\right)_V - P
\ee
Using the right hand side of this equation above, compute the Internal Pressure of a van der Walls gas.

\subsubsection{Solution}
We can re-write the vdw EOS in terms of the molar volume (for convenience). 
\be
P=\frac{RT}{V_m - b}-\frac{a}{V^2_m}
\ee
We begin by computing the required derivative. 
\be
\left(\frac{\pd P}{\pd T}\right)_V = \frac{R}{V_m-b}
\ee
Now we simply substitute in to solve for the internal pressure.
\be
\pi_T = \frac{TR}{V_m-b} - \frac{RT}{V_m - b}+\frac{a}{V^2_m} = \frac{a}{V^2_m} 
\ee

\subsection{Application of the Math}
Calculate $\Delta$U$_m$ for the isothermal expansion of nitrogen gas from an initial volume of 1.00 dm$^3$ to 24.8dm$^3$ at 298K. 

\subsubsection{Solution}
We found in the last calculation that the vdw EOS gives 
\be
\pi_T = \frac{a}{V^2_m} 
\ee
We know that U is a state function with characteristic variables of U(T,V)
Writing the total differential of U we find
\be
dU = \left(\frac{\pd U}{\pd T}\right)_VdT + \left(\frac{\pd U}{\pd V}\right)_TdV 
\ee
We are told the process is done isothermally, therefore we can simplify the differential. 
\be
\begin{split}
dU &= \left(\frac{\pd U}{\pd V}\right)_TdV = \pi_TdV \\
dU_m &= \pi_TdV_m = \frac{a}{V_m^2} dV_m
\end{split}
\ee
Now we simply integrate over our function and run the numbers.
\be
\Delta U_m = \int_{V_i}^{V_f} \frac{a}{V_m^2} dV_m = -\frac{a}{V_f} + \frac{a}{V_i}
\ee
For nitrogen the parameter a = $1.352 \frac{\text{dm}^6-atm}{\text{mol}^2}$.
Substituting in values and converting units I find
\be
\Delta U_m = 131 \frac{J}{\text{mol}}
\ee

\section{Fitting Parameters}

\subsubsection*{Argon}
Consider Argon, whose van der Walls parameters are
\be
a = 1.355 \frac{\text{Bar-}L^2}{\text{mol}^2} \qquad b = 0.03201 \frac{L}{\text{mol}}
\ee
Compute the isothermal compressibility for argon over a range of temperatures (250K to 350K at increments of 10 for example). 
To make life simple set the volume, and moles to 1. 
Make a plot of of $\kappa_T$ vs T for your calculations. 

\subsection{Arbitrary EOS}
Now that we have values for the compressibility see if you can determine a parameter y that is consistent with your previous results determined by the van der Walls EOS. 
Set the pressure, and volume to 1 for convenience. 
Make a plot of the arbitrary parameter y vs T and see what it looks like.

\subsubsection{Solution}
Please see the Github for the Python and Mathematica solutions and comments. 

\end{document}