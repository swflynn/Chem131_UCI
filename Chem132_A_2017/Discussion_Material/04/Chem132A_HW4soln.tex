\documentclass{article}
\usepackage[utf8]{inputenc}

\title{Chem132A Discussion 4 Solutions}
\author{Moises Romero, Shane Flynn}
\date{October 2017}


\usepackage{graphicx}
\usepackage{amsmath}
\usepackage{braket}
\usepackage[margin=0.7in]{geometry}
\usepackage[version=4]{mhchem}


\newcommand{\be}{\begin{equation}}
\newcommand{\ee}{\end{equation}}
\newcommand{\pd}{\partial}

\begin{document}

\maketitle

\section{Thermodynamics of Mixtures}

\subsection{Concepts}
Does a large Henry's Law constant imply a gas is more or less soluble in a liquid (Defend your answer algebraically, and in words)?

\subsection*{Solution}
Henry's Law is given by the following equation
\be
P_a = x_a K
\ee
Where P$_a$ is the pressure of molecule a above the solution, and x$_a$ is the associated mole fraction of a dissolved within the solution.
Consider the following rearrangement. 
\be
\frac{x_a}{P_a} = \frac{1}{K}
\ee
If we increase K, we must subsequently decrease the mole fraction of a in solution to maintain the same a pressure.
This suggests that increasing the Henry constant is associated with decreasing the solubility of a in the liquid phase. 

Physically this means a gas with a larger Henry's Law constant is less soluble in liquid. 

\subsection{Concepts Round Two}
Non-Ideal solutions are usually described through their deviations from Rault's Law. 

\subsection*{Solution}
By definition, a positive deviation occurs when the pure molecules like themselves more than the mixture. 
This means that upon mixing the A-B interaction will necessarily be weaker, pushing the molecules apart. 
The molecules have a weaker attraction and subsequently spend time farther apart increasing the molar volume when compared to the Rault's Law prediction. 

Likewise, if the molecules are weakly attracted upon mixing, it will be easier to completely separate them and the pressure in the gas phase would be larger than predicted by Rault's Law. 
Both of these results are larger than the ideal system, hence the name positive deviations. 

Negative deviations are the exact inverse of this scenario, the pressure and volume of the mixture will be less than anticipated by Rault's Law. 

\subsection{Derivation time}
Ideal (non-simple) solutions are generally defined as either
\be
\mu_i - \mu_{\text{pure},i} = RT\ln x_i
\ee
Or as
\be
f_i = x_i f_{\text{pure},i}
\ee

Show that both of these definitions are consistent (they are the same thing) when I define the fugacity to be:
\be
RT\ln\frac{f_i}{f_i^0} = \mu_i - \mu_i^0
\ee

\subsection*{Solution}
The non-simple solution definitions are both in terms of the pure components, therefore we should take our reference states for the fugacity to be the associated pure states. 
\be
RT\ln\frac{f_i}{f_{\text{pure},i}} = \mu_i - \mu_{\text{pure},i}
\ee
If we substitute this into our first definition we find
\be
RT\ln\frac{f_i}{f_{\text{pure},i}} =  RT\ln x_i \implies \frac{f_i}{f_{\text{pure},i}} = x_i
\ee
Which is exactly equal to the second definition, showing both statements are consistent with this definition of the fugacity. 

\section{Ideal Mixture Calculations}
Consider the mixing of two samples of different mono-atomic ideal gases at two different
temperatures, $T_H$(Temperature hot) and $T_C$ (Temperature cold). 
Initially, one mole of gas A at $T_H$ is enclosed in the left side of a container with volume $\frac{V_a}{2}$. 
While one mole of gas B at $T_C$ is enclosed in the right side, also with volume $\frac{V_a}{2}$ . 
The system is separated from the surroundings by rigid adiabatic walls(q=0). 
The hatch separating the gases is opened and system is allowed to mix. The hatch is then closed.
\subsection{Calculating $T_f$}
Calculate the final temperature in terms of $T_H$ and $T_C$. 
\bigskip

Hint : Recall that for an ideal monoatomic gas : 
\be
C_v = \frac{3R}{2}
\ee
\textbf{Solution:}
 This is an adiabatic process therefore q = 0, and we find U = w. 
 Because the system is closed and we only have ideal gases, the change in internal energy of the system only depends on a change in temperature. 
 But, no heat transfer is allowed, therefore the overall temperature of the system DOES NOT change (the temperature of the two compartments change but the heat transfer is between them). 

We can then write the internal energy of the two gases mixing as : 
\be
\Delta U_{Total}=\Delta U_{A} + \Delta U_{B} = 0 
\ee
For an Ideal gas Internal energy is only a function of temperature : 
\be
\Delta U = \int C_v dT
\ee
We know what $C_v$ is defined as for an ideal gas so we can express equation (6) as follows : 
\be
\Delta U = \int ^{T_f}_{T_H} \frac{3R}{2} + \int ^{T_f}_{T_C} \frac{3R}{2} = 0  
\ee
We then solve the integrals and get : 
\be
\frac{3R}{2}({T_f}-{T_H})  + \frac{3R}{2}({T_f}-{T_C}) = 0 
\ee
We then solve expression (9) for the final temperature and get : 
\be
T_f = \frac{1}{2}(T_H + T_C )
\ee
This result should be intuitive, we would expect the change in temperature to simply be the average. 

\subsection{Calculating the Entropy of the System}

\subsection*{Solution:}
First lets solve for the disorder associated with the temperature redistribution. 
The system is closed, therefore the volume of the system is constant (assume the partition is infinitely small). 
 
Our fundamental equation for the internal energy can be used to calculate the entropy. 
\be
dU=TdS -PdV => dU=TdS 
\ee
Which can be integrated and rearranged as : 
\be \label{eq:1}
\Delta S= \frac{\Delta U}{T} 
\ee
We then want to calculate the entropy change associated with each gas 
\be
\Delta S_{\text{total}}^T = \Delta S_{H} + \Delta S_{C}
\ee

Using equation \ref{eq:1}, we find:
\be
\Delta S^T_{\text{total}} = \frac{\Delta U_H}{T} + \frac{\Delta U_C}{T} = \int^{T_f}_{T_H}{\frac{3R}{2T}}dT + \int^{T_f}_{T_C}{\frac{3R}{2T}}dT = \frac{3R}{2} (\ln\frac{T_f}{T_H} + \ln\frac{T_f}{T_C})
\ee 
This is the entropy associated with the changing in temperature, we must now determine the entropy associated with the different combinations of atoms upon mixing. 

To find the entropy due to the physical mixing we use the fact that we know $\Delta$U = 0.
To calculate entropy we want to consider a reversible pathway (try an isothermal expansion done reversibly): 
\be
q_r= -w = \int P_{ext}dV = \int P_{gas}dV = RT \int VdV = RT \ln\frac{V_f}{V_i}
\ee
We can write our expression for the disorder due to change in volume with : 
\be
\Delta S^V_{\text{total}} = \Delta S_A + \Delta S_B 
\ee
Although we close the latch, initially each gas is restricted to half of the box. 
Once we allow them to mix each gas can occupy the entire box. 
Opening and closing the latch does not effect this process, each gas is able to double its volume (we are now looking at the two subsystems in the problem). 
\be
q_r = -w = RT \ln\frac{V_f}{V_i} = RT \ln\frac{2V_i}{V_i} = RT \ln 2 
\ee
This finds us a reversible heat that we can substitute into the definition of Entropy. 

\be
\Delta S^V_{total} = \frac{RT \ln 2}{T} + \frac{RT \ln 2}{T} = 2R \ln 2
\ee
The total disorder is just the sum of the teo pathways we constructed: Temperature and Volume: 
\be
\Delta S_{\text{Mixture}} = \Delta S^T_{\text{total}} + \Delta S^V_{\text{total}} = \frac{3R}{2} (\ln\frac{T_f}{T_H} + \ln\frac{T_f}{T_C}) + 2R \ln 2
\ee

\section{van der Walls Phase Diagram}
This is a very famous problem, and the values below the critical temperature are non-physical.
This was addressed by Maxwell (Maxwell Constructions) and can be used to determine the phase diagram for a van der Walls fluid. 
You will most probably work through the mathematics of determining the critical values given to you above, and the Maxwell construction during your undergraduate quantum mechanics courses. 
See the github page for the Python and Mathematica solutions. 

\end{document}