\documentclass{article}
\usepackage[utf8]{inputenc}

\title{Chem132A Discussion 4 Solutions}
\author{Moises Romero, Shane Flynn}
\date{October 2017}


\usepackage{graphicx}
\usepackage{amsmath}
\usepackage{braket}
\usepackage[margin=0.7in]{geometry}
\usepackage[version=4]{mhchem}


\newcommand{\be}{\begin{equation}}
\newcommand{\ee}{\end{equation}}
\newcommand{\pd}{\partial}

\begin{document}

\maketitle

\section{Thermodynamics of Mixtures}

\subsection{Concepts}
Does a large Henry's Law constant imply a gas is more or less soluble in a liquid (Defend your answer algebraically, and in words)?

\subsection*{Solution}
Henry's Law is given by the following equation
\be
P_a = x_a K
\ee
Where P$_a$ is the pressure of molecule a above the solution, and x$_a$ is the associated mole fraction of a dissolved within the solution.
Consider the following rearrangement. 
\be
\frac{x_a}{P_a} = \frac{1}{K}
\ee
If we increase K, we must subsequently decrease the mole fraction of a in solution to maintain the same a pressure.
This suggests that increasing the Henry constant is associated with decreasing the solubility of a in the liquid phase. 

Physically this means a gas with a larger Henry's Law constant is less soluble in liquid. 

\subsection{Concepts Round Two}
Non-Ideal solutions are usually described through their deviations from Rault's Law. 

\subsection*{Solution}
By definition, a positive deviation occurs when the pure molecules like themselves more than the mixture. 
This means that upon mixing the A-B interaction will necessarily be weaker, pushing the molecules apart. 
The molecules have a weaker attraction and subsequently spend time farther apart increasing the molar volume when compared to the Rault's Law prediction. 

Likewise, if the molecules are weakly attracted upon mixing, it will be easier to completely separate them and the pressure in the gas phase would be larger than predicted by Rault's Law. 
Both of these results are larger than the ideal system, hence the name positive deviations. 

Negative deviations are the exact inverse of this scenario, the pressure and volume of the mixture will be less than anticipated by Rault's Law. 

\subsection{Derivation time}
Ideal (non-simple) solutions are generally defined as either
\be
\mu_i - \mu_{\text{pure},i} = RT\ln x_i
\ee
Or as
\be
f_i = x_i f_{\text{pure},i}
\ee

Show that both of these definitions are consistent (they are the same thing) when I define the fugacity to be:
\be
RT\ln\frac{f_i}{f_i^0} = \mu_i - \mu_i^0
\ee

\subsection*{Solution}
The non-simple solution definitions are both in terms of the pure components, therefore we should take our reference states for the fugacity to be the associated pure states. 
\be
RT\ln\frac{f_i}{f_{\text{pure},i}} = \mu_i - \mu_{\text{pure},i}
\ee
If we substitute this into our first definition we find
\be
RT\ln\frac{f_i}{f_{\text{pure},i}} =  RT\ln x_i \implies \frac{f_i}{f_{\text{pure},i}} = x_i
\ee
Which is exactly equal to the second definition, showing both statements are consistent with this definition of the fugacity. 

\section{Ideal Mixture Calculations}
Consider the mixing of two samples of different mono-atomic ideal gases at two different
temperatures, $T_H$(Temperature hot) and $T_C$ (Temperature cold). 
Initially, one mole of gas A at $T_H$ is enclosed in the left side of a container with volume $\frac{V_a}{2}$. 
While one mole of gas B at $T_C$ is enclosed in the right side, also with volume $\frac{V_a}{2}$ . 
The system is separated from the surroundings by rigid adiabatic walls(q=0). 
The hatch separating the gases is opened and system is allowed to mix. The hatch is then closed.
\subsection{Calculating $T_f$}
Calculate the final temperature in terms of $T_H$ and $T_C$. 
\bigskip

Hint : Recall that for an ideal monoatomic gas : 
\be
C_v = \frac{3R}{2}
\ee
\textbf{Solution:}
 This is an adiabatic process therefore q= 0 , this also a free expansion of an Ideal Gas therefore w=0.  
We can then write the internal energy of the two gases mixing as : 
\be
\Delta U_{Total}=\Delta U_{A} + \Delta U_{B} = 0 
\ee
For an Ideals gas Internal energy is only a function of temperature : 
\be
\Delta U = \int C_v dT
\ee
We know what $C_v$ is defined as for an ideal gas so we can express equation (6) as follows : 
\be
\Delta U = \int ^{T_f}_{T_H} \frac{3R}{2} + \int ^{T_f}_{T_C} \frac{3R}{2} = 0  
\ee
We then solve the integrals and get : 
\be
\frac{3R}{2}({T_f}-{T_H})  + \frac{3R}{2}({T_f}-{T_C}) = 0 
\ee
We then solve expression (9) for the final temperature and get : 
\be
T_f = \frac{1}{2}(T_H + T_C )
\ee

\subsection{Calculating the Entropy of the System}
Hint: You will want calculate disorder for the change in temperature to reach $T_f$ and a disorder to account for the change in volume. 
\textbf{Solution:}
First lets solve for the disorder associated with temperature. 
For this closed system there is a constant volume as each gas occupies 1/2 the volume of the total system so when the adiabatic wall is removed there is no change in volume of the total system. 
We can thus derive the following expression : 
\be
dU=TdS -PdV => dU=TdS 
\ee
Which can be integrated and rearranged as : 
\be
S= \frac{U}{T}
\ee
For the temperature for the Hot and Cold gas the total entrop disorder can be expressed as " 
\be
\Delta S^T_{total} = \Delta S_{H} + \Delta S_{C}
\ee
We can then sub expression (12) into (13) to get : 
\be
\Delta S^T_{total} = \frac{U_H}{T} + \frac{U_C}{T} = \int^{T_f}_{T_H}{\frac{3R}{2T}} + \int^{T_f}_{T_C}{\frac{3R}{2T}} = \frac{3R}{2} (\ln\frac{T_f}{T_H} + \ln\frac{T_f}{T_C})
\ee 

In order to find the disorder created due to change in volume of the individual gases we use the fact that we know U = 0 and consider a reversible pathway to write : 
\be
q_r= -w = \int P_{ext}dV = \int P_{gas}dV = RT \int VdV = RT \ln\frac{V_f}{V_i}
\ee
We can write our expression for the disorder due to change in volume with : 
\be
\Delta S^V_{total} = \Delta S_H + \Delta S_C 
\ee
We know that the relationship between initial volume and final volume is as folows : 
\be
V_i = \frac{V_f}{2} => V_f = 2Vi 
\ee
We can then sub this in for our expression for heat and work : 
\be
q_r = -w = RT \ln\frac{V_f}{V_i} = RT \ln\frac{2V_i}{V_i} = RT \ln 2 
\ee
We can know sub in our reversible heat expression to the disorder due to volume expression : 
\be
\Delta S^V_{total} = \frac{RT \ln 2}{T} + \frac{RT \ln 2}{T} = 2R \ln 2
\ee
The total disorder of the mixture of ideal gases is expressed as a sum of the disorder due to Temperature and Volume : 
\be
\Delta S_{Mixture} = \Delta S^T_{total} + \Delta S^V_{total} = \frac{3R}{2} (\ln\frac{T_f}{T_H} + \ln\frac{T_f}{T_C}) + 2R \ln 2
\ee

\section{van der Walls Phase Diagram}
For the vdw Equation of State, a bunch of mathematics can be used to determine the critical temperature, pressure, and volume as
\be
V_{m,c} = 3b, \qquad P_c = \frac{a}{27b^2}, \qquad T_c = \frac{8a}{27bR}
\ee

Consider Krypton as described by the vdw EOS. 
Compute the critical pressure, temperature, and molar volume for Krypton using the above expressions.  

Once you have computed these values, make three separate isotherms (PV diagrams holding T constant) where you plot values  above the critical temperature, below the critical temperature, and at the critical temperature. 

This is a very famous problem, and the values below the critical temperature are non-physical.
This was addressed by Maxwell (Maxwell Constructions) and can be used to determine the phase diagram for a van der Walls fluid. 
You will most probably work through the mathematics of determining the critical values given to you above, and the Maxwell construction during your undergraduate quantum mechanics courses. 


\end{document}