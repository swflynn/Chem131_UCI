\documentclass{article}
\usepackage[utf8]{inputenc}

\title{Chem132A Discussion 3 Solutions}
\author{Moises Romero (moiseser@uci.edu), Shane Flynn (swflynn@uci.edu) }
\date{10/9/17}


\usepackage{graphicx}
\usepackage{amsmath}
\usepackage{braket}
\usepackage[margin=0.7in]{geometry}


\newcommand{\be}{\begin{equation}}
\newcommand{\ee}{\end{equation}}
\newcommand{\pd}{\partial}

\begin{document}

\maketitle

\section{Entropy}

\subsection{Pathway A: Solution}
Compute the Entropy for: 1 mole of gas traversing pathway A.
Take path A to be an isothermal expansion. 

First: our equation of state can be simplified: 
\be
P = RTx + RTy
\ee
Pathway A is an isothermal expansion so dU$_{A}$=0 
It is a reversible expansion so we can write the following expression:   
\be 
0 = q_{r,A} + w_{r,A} \gg  q_{r}=-w_{r} = \int P_{gas}dV = \int ({RTx}+{RTy}) dV
\ee

Since we have an expression for q$_{rev}$ we can solve for the Entropy of pathway A. 
\be
\Delta S_{A} = \frac{\delta q_{r}}{T}= \int \frac{RT(x+y)}{T} dV = R \int x+y dV = R (x\Delta V + y\Delta V )
\ee

\subsection{Pathway B+C: Solution}
Next, compute the Entropy for 1 mole of gas traversing pathways B and then C.  
Take path B to be a reversible adiabatic expansion and path C a reversible isochoric process. 

\textbf{Hint:}
For pathway B you will need to transform your integral from dT to dP to make a meaningful comparison. 

Pathway B is a reversible adiabatic expansion , so q$_{r,B}$ = 0 therefore : 

\be
\Delta S_{B} = 0
\ee

Pathway C is an isochoric process so w$_{r,C}$ = 0 so : 

\be 
\delta q_{r,C} = dU_C = C_v(T)dT 
\ee

So the expression for heat is : 
\be
q_{r,C} = \int_{T_2}^{T_1} C_v(T)dT
\ee

So our expression for entropy is : 
\be
\Delta S_C = \int_{T_2}^{T_1}\frac{ C_v(T)dT}{T} = -\int_{T_1}^{T_2} \frac{ C_v(T)dT}{T}
\ee

\textbf{Note:} We reverse the bounds on this integral in order to relate it to the next step. 

With this we can develop an expression for 
Pathway B + C to reach state 3 : 

\be
\Delta S_{B+C} = \Delta S_B + \Delta S _C = -\int_{T_1}^{T_2} \frac{ C_v(T)dT}{T}
\ee

Looking at pathway B again we can relate a change in temperature to a change in volume : 
\be
\begin{split}
\Delta U &= w_{r} \\
dU &\propto C_V dT\\
\delta w &\propto -PdV \\ 
dT &\propto dV
\end{split}
\ee
Using this relationship we can convert our integral for entropy into a similar form found in pathway A.

\be
\int_{T_1}^{T_2} { C_v(T)dT} = \int_{V_1}^{V_2} -P_{gas}dV = -RT(x\Delta V + y\Delta V )
\ee

We can then divide this expression by T to create the following : 
\be
\int_{T_1}^{T_2} \frac{ C_v(T)dT}{T} =\int_{V_1}^{V_2} \frac{-P_{gas}dV}{T} = -R (x\Delta V + y\Delta V )
\ee

We Substitute this new expression into our Entropy expression at (11) to get : 

\be
\Delta S_{B+C} = R (x\Delta V + y\Delta V ) 
\ee

And we can see that regardless of the heats not being equal we get : 

\be
\Delta S_{A}= \Delta S_{B+C} 
\ee

\subsection{Time to be Lazy: Solution}

Because Entropy is a state function it must be equal to that of any pathway with the same start and end conditions. 

\subsection{Entropy of a Free Expansion: Solution}

\be
\frac{P}{V}=\frac{RTx}{V}+\frac{RTy}{V}
\ee

The Entropy equation uses reversible heat therefore we need to consider a reversible pathway. 
Since U is only a function of temperature, dU = 0 therefore : 

\be
q_{rev}=-w_{r}= \int P_{gas}dV = RT(x\Delta V + y\Delta V) 
\ee

So we can express the Entropy as : 
\be
\Delta S = \frac{RT\int_{V_1}^{V_2} (x+y)(dV)}{T} = R(x\Delta V + y\Delta V) 
\ee

As expected the entropy needs to increase with the expansion of the gas into the vacuumn. 

\section{Fundamental Equations}

\subsection{Internal Energy and Entropy}
Consider a closed system, with an ideal gas undergoing a reversible expansion (assume only PV work). 

\subsubsection{New Variables: Solution}

For a reversible process we can write the Entropy as an equality, $dS =\frac{\delta q_r}{t} \rightarrow TdS = \delta q_r$. 
If we only consider PV work we can write $\delta w = -PdV$. 
\begin{equation}
    \begin{split}
        dU &= \delta q + \delta w \\
        dU &= TdS - PdV \\
    \end{split}
\end{equation}
This is a general expression combining the first two laws of thermodynamics (it can be generalized to other work and more complications but ignore all that). 
It tells us that when we are interested in changing the internal energy we 'naturally' make changes in the entropy and volume. 

Notice this equation only contains state functions as variables (the things that are changing).
This is very important, although we derived the relationship for a reversible process, all the variables changing are state functions, and therefore are path independent. 
 
We have now shown that U $\rightarrow$ U(S,V). 

\subsubsection{Partial Derivatives: Solution}

\begin{equation}
\begin{split}
    dU &= \left( \frac{\partial U}{\partial S} \right)_VdS + \left( \frac{\partial U}{\partial V}  \right)_S dV\\
     dU &= TdS - PdV \\
    \left( \frac{\partial U}{\partial S} \right)_V &= T \\
    \left( \frac{\partial U}{\partial V}  \right)_S &= -P
\end{split}
\end{equation}
We know both equations must be true, therefore we simply equate terms. 

\subsubsection{More Math: Solution}
\begin{equation}
\begin{split}
\frac{\partial ^2 U}{\partial S \partial V} &= \frac{\partial ^2 U}{\partial V \partial S} \\
\frac{\partial }{\partial S}\left[\left(\frac{\partial U}{\partial V}\right)_S\right]_V &= \frac{\partial}{\partial V}\left[\left(\frac{\partial U }{\partial S}\right)_V\right]_S \\
-\left(\frac{\partial P}{\partial S}\right)_V &= \left(\frac{\partial T}{\partial V}\right)_S
\end{split}
\end{equation}
This final relationship is also know as a \textbf{Maxwell Relationship}.
All of these math operations can be repeated for each \textbf{Thermodynamic Potential} to produce new relationships. 

I encourage you to understand the process, actually deriving these relationships should only take a few minutes at most. 

\subsection{Enthalpy and The First 2 Laws}

\subsubsection{The Fundamental Equation: Solution}
\be
    \begin{split}
        H &\equiv U + PV \\
        dH &= dU + d(PV) = dU + PdV + VdP \\
        dH &= TdS - PdV + PdV + VdP \\
        dH &= TdS + VdP
    \end{split}
\ee
And again we see that entropy and pressure are natural variables of the entropy: H(S,P). 

\subsubsection{The Total Differential}
\be
\begin{split}
    dH &= \left( \frac{\partial H}{\partial S} \right)_PdS + \left( \frac{\partial H}{\partial P}  \right)_S dP\\
     dH &= TdS + VdP \\
    \left( \frac{\partial H}{\partial S} \right)_P &= T \\
    \left( \frac{\partial H}{\partial P}  \right)_S &= V
\end{split}
\ee

\subsubsection{Computing Your Maxwell: Solution}

\be
\begin{split}
\frac{\partial ^2 H}{\partial S \partial P} &= \frac{\partial ^2 H}{\partial P \partial S} \\
\frac{\partial }{\partial S}\left[\left(\frac{\partial H}{\partial P}\right)_S\right]_P &= \frac{\partial}{\partial P}\left[\left(\frac{\partial H }{\partial S}\right)_P\right]_S \\
 \left(\frac{\partial V}{\partial S}\right)_P &=
\left(\frac{\partial T}{\partial P}\right)_S
\end{split}
\ee

\subsection{Helmholtz and The First 2 Laws}

\subsubsection{The Fundamental Equation: Solution}
\be
    \begin{split}
        A &\equiv  U - TS \\
        dA &= dU - d(TS) = dU -TdS - SdT \\
        dA &= TdS - TdS - PdV - SdT \\
        dA &= -PdV - SdT
    \end{split}
\ee
A(V,T)

\subsubsection{The Total Differential: Solution}
\be
\begin{split}
    dA &= \left( \frac{\partial A}{\partial V} \right)_T dV + \left( \frac{\partial A}{\partial T}  \right)_V dT\\
     dA &= -PdV -SdT \\
    \left( \frac{\partial A}{\partial V} \right)_T &= -P \\
    \left( \frac{\partial A}{\partial T}  \right)_V &= -S
\end{split}
\ee

\subsubsection{Computing Your Maxwell}
\be
\begin{split}
\frac{\partial ^2 A}{\partial T \partial V} &= \frac{\partial ^2 A}{\partial V \partial T} \\
\frac{\partial }{\partial T}\left[\left(\frac{\partial A}{\partial V}\right)_T\right]_V &= \frac{\partial}{\partial V}\left[\left(\frac{\partial A }{\partial T}\right)_V\right]_T \\
\left(\frac{\partial P}{\partial T}\right)_V &= \left(\frac{\partial S}{\partial V}\right)_T
\end{split}
\ee

\subsection{Gibbs as a Double Transformation}

\subsubsection{The Fundamental: Solution}
\be
\begin{split}
G &\equiv A + PV \\
dG &= dA + PdV + VdP \\
dG &= -SdT -PdV + PdV + VdP\\
dG &= -SdT + VdP
\end{split}
\ee

\subsubsection{The Total Dif: Solution}
\be
\begin{split}
    dG &= \left( \frac{\partial G}{\partial T} \right)_PdT + \left( \frac{\partial G}{\partial P}  \right)_T dP\\
     dG &= -SdT + VdP \\
    \left( \frac{\partial G}{\partial T} \right)_P &= -S \\
    \left( \frac{\partial G}{\partial P}  \right)_T &= V
\end{split}
\ee

\subsubsection{Computing Your Maxwell: Solution}
\begin{equation}
\begin{split}
\frac{\partial ^2 G}{\partial T \partial P} &= \frac{\partial ^2 G}{\partial P \partial T} \\
\frac{\partial }{\partial T}\left[\left(\frac{\partial G}{\partial P}\right)_T\right]_P &= \frac{\partial}{\partial P}\left[\left(\frac{\partial G }{\partial T}\right)_P\right]_T \\
\left(\frac{\partial V}{\partial T}\right)_P &= -\left(\frac{\partial S}{\partial P}\right)_T
\end{split}
\end{equation}

\end{document}