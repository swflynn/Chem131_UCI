\documentclass{article}
\usepackage[utf8]{inputenc}

\title{Chem132A Discussion 7 Solutions}
\author{Moises Romero, Shane Flynn}
\date{November 2017}


\usepackage{graphicx}
\usepackage{amsmath}
\usepackage{braket}
\usepackage[margin=0.7in]{geometry}
\usepackage[version=4]{mhchem}


\newcommand{\be}{\begin{equation}}
\newcommand{\ee}{\end{equation}}
\newcommand{\pd}{\partial}

\begin{document}

\maketitle
\section{Chemical Kinetics}
\subsection*{Reaction to Consider}
The reaction for the decomposition of F$_2$O is : 

\be
2F_2O(g) \rightarrow 2F_2(g) + O_2(g)
\ee

After various experiments, the following reaction mechanism was published :  

\be
(1) \quad F_2O + F_2O \rightarrow  F + OF + F_2O \quad k_a 
\ee

\be
(2) \quad F + F_2O \rightarrow F_2 + OF \quad k_b 
\ee

\be 
(3) \quad OF + OF \rightarrow O_2 + F + F \quad k_c
\ee

\be
(4) \quad F + F + F_2O \rightarrow F_2 + F_2O \quad k_d
\ee

\subsection{Steady-State Approximation}

Using the steady-state approximation show that the mechanism follows the experimental rate law of : 
\be
\frac{-d[F_2O]}{dt} = k_r[F_2O]^2 + k'_r[F_2O]^{\frac{3}{2}}
\ee

Recall that the steady state approximation assumes that intermediate, \textbf{I} , has a constant concentration.   
\be
\frac{d[I]}{dt} = 0 
\ee

\textbf{Hint}: For the decomposition the intermediate products are F and OF. 

\subsection*{Solution:}
I will start by writing down all of the rates associated with each step in the mechanism. 
Because the mechanism is written with only forward arrows, we can simply define the rates in terms of the products. 
\be
\begin{split}
    \text{Rate}_1 &= k_a[F_2O]^2\\
    \text{Rate}_2 &= k_b[F][F_2O] \\
    \text{Rate}_3 &= k_c[OF]^2 \\
    \text{Rate}_4 &= k_d[F]^2[F_2O] \\
\end{split}
\ee

The overall problem is to determine the rate law for F$_2$O, to do this we can look at all of the steps within the mechanism that generate/consume F$_2$O. 

\bigskip

\textbf{RECALL}: This is the overall rate of a molecule in the original equation. 
We can use any reactant or product to define the overall rate, but we must account for the stoichometry in this step!
For example: if the reaction were A + 2B $\ce{->}$ 3C + D, then we know
\be
\frac{1}{3}\frac{d}{dt}[C] = \frac{d}{dt}[D] = -\frac{d}{dt}[A] = -\frac{1}{2}\frac{d}{dt}[C]
\ee
In the same way we will write the overall rate in terms of F$_2$O. 
\be
\frac{1}{2}\frac{d}{dt}[F_2O] = -k_a[F_2O]^2 - k_b[F][F_2O] 
\ee
Where step 1 of the mechanism is overall negative wrt F$_2$O and step 4 is overall neutral and does not play a factor in the rate. 
Again the RHS (the individual rates) do not have any stoichometry included it is accounted for by the rate law itself (raising the concentrations to various powers). 
This result looks similar to the rate law we are trying to prove, however, it is in terms of the intermediate F.

To remove this intermediate dependence we make a steady state approximation and write down the rate of formation of the intermediates in a similar manner. 
Please note: we are no longer writing rates of an overall reaction, these are rates associated with the mechanism, therefore no coefficients are used, simply the rate law. 

From the mechanism we see that F is formed in steps 1/3 , and F is consumed in steps 2/4. 
\be
\frac{d[F]}{dt} = 0 = k_a [F_2O]^2 - k_b[F][F_2O] + k_c[OF]^2 -k_d[F]^2[F_2O]
\ee

We will then write down the steady state approximation for the other intermediate in the same way. 
\be
\frac{d[OF]}{dt} = 0 = k_a[F_2O]^2 + k_b[F][F_2O] - k_c[OF]^2 
\ee

Remember, our goal was to remove the [F] intermediate in our overall rate of F$_2$O.
But the intermediate F and the intermediate OF both depend on the same variables.
We can therefore take the equations and combine them to reduce the overall number of variables (we have 2 intermediates, we therefore need 2 equations to solve). 

If we add each intermediate equation together we see some nice cancellations! 
\be
2k_a[F_2O]^2 - k_d[F]^2[F_2O] = 0 
\ee

We will solve this equation for [F], so that we can substitute into our overall rate law. 
\be
[F] = \left(2\frac{k_a}{k_d}[F_2O]\right)^{\frac{1}{2}}
\ee

Finally we substitute in our expression for [F] into the overall rate law of F$_2$O.  
Doing the algebra I find:
\be
\begin{split}
\frac{1}{2}\frac{d}{dt}[F_2O] &= -k_a[F_2O]^2 - k_b\left(2\frac{k_a}{k_d}[F_2O]\right)^{\frac{1}{2}} [F_2O] \\
-\frac{d}{dt}[F_2O] &= 2k_a[F_2O]^2 + 2k_b\left(2\frac{k_a}{k_d}\right)^{\frac{1}{2}} [F_2O]^{3/2} \\
\end{split}
\ee

\subsection{Arrhenius Equation}
The Arrhenius equation allows us to determine the activation energies of elementary reactions. 
\be
\ln k = \ln A -\frac{E}{RT}
\ee
For the decomposition determine the activation energy for elementary equation 2 in the provided mechanism. 
 The experimentally determined Arrhenius parameters are as follows for the temperature range 501-583K: 

\bigskip

For k$_r$ A=7.8 x 10$^{13}$dm$^3$mol$^{-1}$s$^{-1}$ and E$_a$ = 1.935 x 10$^4$KR. 

\bigskip

For k'$_r$ A=2.3 x 10$^{10}$dm$^3$mol$^{-1}$s$^{-1}$ and E$_a$ = 1.691 x 10$^4$KR. 

Where R is the gas constant.  
For the k$_d$ term in the Arrhenius equation can be ignored as step 4 in the reaction mechanics is termolecular. This term means a reaction involving three molecules colliding which has a much smaller probability of occurring than two molecules colliding. Thus this term can be ignored. 
\subsection*{Solution:}
We want to solve for E$_b$ , so we will want solve for k$_b$. We know the expression for k'$_r$ contains k$_b$. 
We will want to take that expression and multiply by ln , and then  separate our logs to simplify more..
\be
\ln k'_r = \ln \left[k_b\left(\frac{k_a}{k_d}\right)\right] = \ln k_b +\frac{1}{2}\ln k_a - \frac{1}{2}\ln k_d 
\ee
Recall that step 4 of the mechanism is termolecular(probability of 3 compounds colliding is very small) therefore k$_d$ term can be ignored. From our derivation in the first problem we know k$_r$ = k$_a$. So the expression is now : 
\be
\ln k'_r = \ln k_b +\frac{1}{2}\ln k_r
\ee
We can rewrite using the Arrhenius equation as : 
\be
\ln A_r'- \frac{E_r'}{RT}=\ln A_b - \frac{E_b}{RT} + \frac{1}{2}\ln A_r - \frac{1}{2}\frac{E_r}{RT}
\ee
We can take the derivative with respect to Temperature to get the activation energies. 
\be
\frac{E_r'}{RT^2}=\frac{E_b}{RT^2} +   \frac{1}{2}\frac{E_r}{RT^2}
\ee

We can multiply through by RT$^2$ and rearrange to solve for E$_b$
\be
E_b = E'_r - \frac{1}{2}E_r = 1.691 \times 10^4\text{K}R-(\frac{1}{2})1.935 \times 10^4\text{K}R = 7235K(R) = (7235K)(.008314\text{kJmol}K^{-1}) = 60\text{kJmol}^{-1} 
\ee


\section{Michaelis-Menten}
Some of the most important biological reaction are the class of \textbf{Enzyme-Catalyzed Reactions}. 
Experimentally it has been shown that these reactions usually follow a rate law of the form
\be
-\frac{d}{dt}[S] = \frac{k[S]}{K + [S]}
\ee
In 1913, Leonor Michaelis and Maude Menten showed that  a simple mechanism could account for this rate law (in terms of as Enzyme, and a Substrate). 
\be\label{MM}
E + S \underset{k_{-1}}{\stackrel{k_1}{\rightleftharpoons}} ES \underset{k_{-2}}{\stackrel{k_2}{\rightleftharpoons}} E + P
\ee

In this question we will explore various applications of the model to understand biochemistry. 

\subsection{Traditional Equation}
To start, we will derive the traditional M-M Equation. 

\subsubsection{Dissociation}
What reaction will give the following rate constant, and what doe sit represent physically?
\be
k_d = \frac{[E][S]}{[ES]}
\ee

\subsubsection*{Solution}
We know that our rates are written in terms of products divided by reactions, therefore. 
\be
ES \rightleftharpoons E + S
\ee
This is simply the reverse reaction in step 1 of Equation \ref{MM}.
This reaction is the  dissociation of the enzyme-substrate complex into separate enzyme and substrate (ie. k$_d$ = k$_{-1}$).

\subsubsection{Rate-Determining step}
Assuming the Enzyme-Substrate Generating product is the rate determining step, what is the rate of product formation?

\subsubsection*{Solution}
If we assume the second half of equation \ref{MM} is the rate-determining step (i.e. it is irreversible), than the rate of product formation is simply controlled by this equation. 
\be
r_p = k_2[ES]
\ee

\subsubsection{Enzyme Conservation}
Determine the concentration of the enzyme-substrate complex by assuming total amount of enzyme [E]$_T$ is conserved. 

\subsubsection*{Solution}
If the total  amount of enzyme is conserved, and  the enzyme can only be part of the complex or alone we can write
\be
[E]_T = [E] + [ES]
\ee
Now using the dissociation process above we can substitute in an expression for [E].
\be
\begin{split}
[E]_T &= \frac{k_d[ES]}{[S]} + [ES] = [ES] \left(\frac{k_d}{[S]}+1\right)\\
[ES] &= \frac{[E]_T}{\frac{k_d}{[S]}+1}
\end{split}
\ee

\subsubsection{Simplify}
Show that the rate of product formation (r$_p$) is given by 
\be
r_p = \frac{v_{max}[S]}{[S] + k_d}
\ee
Where v$_{max}$ is defined to be k$_2$[E]$_T$. 

\subsubsection*{Solution}
We just need to substitute in our previous result for r$_p$ and solve.
\be
\begin{split}
r_p &= k_2 \left(\frac{[E]_T}{\frac{k_d}{[S]}+1}\right) = \frac{v_{max}}{\frac{k_d}{[S]}+1}\\
r_p &= \frac{v_{max}[S]}{k_d + [S]}
\end{split}
\ee

\subsection{Uncompetitive Inhibition}
We will now investigate a simple twist to the model, to account for an inhibitor.
In general an inhibitor is a molecule that can bind to an enzyme and decrease its activity. 
An \textbf{Uncompetitive Inhibitor}  is a molecule that can only bind to an Enzyme-Substrate complex. 

\subsubsection{Mechanism}
Add a new equation to account for the Inhibitor, assuming it establishes an equilibrium, write down the dissociation constant for the inhibitor (k$_i$). 

\subsubsection*{Solution}
As stated, the inhibitor only interacts with the ES complex, therefore 
\be
[ES] + [I] \rightleftharpoons [ESI]
\ee
We can then write the dissociation of the complex as 
\be
k_I = \frac{[ES][I]}{[ESI]}
\ee

\subsubsection{Enzyme Concentration}
Find an expression for the enzyme concentration.
State all assumptions along the way. 

\subsubsection*{Solution}
If we assume a low concentration of the product concentration, than the rate-determining step becomes irreversible. 
If we then assume the total enzyme concentration is conserved we can write.
\be
[E]_T = [E] + [ES] + [ESI]
\ee
We can now substitute in our dissociation expression.
\be
\begin{split}
[E]_T &= [E] + [ES] + [ESI]\\
[E]_T &= [E] + \frac{[E][S]}{k_d} + \frac{[ES][I]}{k_I}\\
[E]_T &= [E] + \frac{[E][S]}{k_d} + \frac{[E][S][I]}{k_I}\\
\end{split}
\ee
Finally we can factor out terms and solve for the concentration of the enzyme. 
\be
\begin{split}
[E]_T &= [E]\left(1+\frac{[S]}{k_d} + \frac{[S][I]}{k_dk_I}\right)\\
[E] &= \frac{[E]_T}{\left(1+\frac{[S]}{k_d} + \frac{[S][I]}{k_dk_I}\right)}
\end{split}
\ee

\subsubsection{Product Formation Rate}
Determine the rate of formation of the products (simplify as much as possible).

\subsubsection*{Solution}
Again we start with r$_p$ and simplify.
\be
\begin{split}
r_p &= k_2[ES] \\
&= k_2\frac{[E][S]}{k_d}\\
&= k_2\left[\frac{[E]_T[S]}{k_d \left(1+\frac{[S]}{k_d} + \frac{[S][I]}{k_dk_I}\right)}\right]\\
r_p &= \frac{v_{max}[S]}{k_d + [S]\left(1+\frac{[I]}{k_i}\right)}
\end{split}
\ee

And if we compare this to our original M-M expression for the rate, the inhibitor will make this process slower.







\end{document}