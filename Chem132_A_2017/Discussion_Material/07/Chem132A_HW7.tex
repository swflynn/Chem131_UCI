\documentclass{article}
\usepackage[utf8]{inputenc}

\title{Chem132A Discussion  Homework}
\author{Moises Romero (moiseser@uci.edu), Shane Flynn (swflynn@uci.edu) }
\date{12/3/17}

\usepackage{graphicx}
\usepackage{amsmath}
\usepackage{braket}
\usepackage[margin=0.7in]{geometry}
\usepackage[version=4]{mhchem}


\newcommand{\be}{\begin{equation}}
\newcommand{\ee}{\end{equation}}
\newcommand{\ben}{\begin{equation*}}
\newcommand{\een}{\end{equation*}}
\newcommand{\pd}{\partial}

\begin{document}

\maketitle

\section{Chemical Kinetics}
Chemical Kinetics is the study of reaction rates.
The mathematics and experimental findings associated with chemical reactions can be combined to better understand potential mechanisms through which the reaction can occur. 

\subsection*{Reaction to Consider}
The reaction for the decomposition of F$_2$O is : 

\be
2F_2O(g) \rightarrow 2F_2(g) + O_2(g)
\ee

After various experiments, the following reaction mechanism was published : 
\ben
(1) \quad F_2O + F_2O \rightarrow  F + OF F_2O \quad k_a 
\een
\ben
(2) \quad F + F_2O \rightarrow F_2 + OF \quad k_b 
\een
\ben
(3) \quad OF + OF \rightarrow O_2 + F + F \quad k_c
\een
\ben
(4) \quad F + F + F_2O \rightarrow F_2 + F_2O \quad k_d
\een

\subsection{Steady-State Approximation}

Using the steady-state approximation show that the mechanism follows the experimental rate law of : 
\be
\frac{-d[F_2O]}{dt} = k_r[F_2O]^2 + k'_r[F_2O]^{\frac{3}{2}}
\ee

Recall: the steady state approximation assumes that the intermediate, \textbf{I} , has a constant concentration:
\be
\frac{d[I]}{dt} = 0 
\ee

\textbf{Hint}: For the decomposition, the intermediate products are F and OF. 

\subsection{Arrhenius Equation}
The Arrhenius equation allows us to determine the activation energies of elementary reactions. 
\be
\ln k_r = \ln A -\frac{E_a}{RT}
\ee
For the decomposition determine the activation energy for equation 2 in the provided mechanism. 

The experimentally determined Arrhenius parameters are as follows (for the temperature range 501-583 K): 

\bigskip

For k$_r$ A=7.8 x 10$^{13}$dm$^3$mol$^{-1}$s$^{-1}$ and $\frac{E_a}{R}$ = 1.935 x 10$^4$K. 

\bigskip

For k'$_r$ A=2.3 x 10$^{10}$dm$^3$mol$^{-1}$s$^{-1}$ and $\frac{E_a}{R}$ = 1.691 x 10$^4$K. 

\section{Michaelis-Menten}
Some of the most important biological reaction are the class of \textbf{Enzyme-Catalyzed Reactions}. 
Experimentally it has been shown that these reactions usually follow a rate law of the form
\be
-\frac{d}{dt}[S] = \frac{k[S]}{K + [S]}
\ee
In 1913, Leonor Michaelis and Maude Menten showed that  a simple mechanism could account for this rate law (in terms of as Enzyme, a Product, and a Substrate). 
\be\label{MM}
E + S \underset{k_{-1}}{\stackrel{k_1}{\rightleftharpoons}} ES \underset{k_{-2}}{\stackrel{k_2}{\rightleftharpoons}} E + P
\ee

In this question we will explore this model to understand biochemistry. 

\subsection{Traditional Equation}
To start, we will derive the traditional M-M Equation. 

\subsubsection{Dissociation}
What reaction will give the following rate constant, and what does it represent physically?
\be
k_d = \frac{[E][S]}{[ES]}
\ee

\subsubsection{Rate-Determining step}
Assuming the Enzyme-Substrate complex forming the product is the rate determining step, what is the rate of product formation?

\subsubsection{Enzyme Conservation}
Determine the concentration of the enzyme-substrate complex by assuming total amount of enzyme [E]$_T$ is conserved. 

\subsubsection{Simplify}
Show that the rate of product formation (r$_p$) is given by 
\be
r_p = \frac{v_{max}[S]}{[S] + k_d}
\ee
Where v$_{max}$ is defined to be k$_2$[E]$_T$. 

\subsection{Uncompetitive Inhibition}
We will now investigate a simple twist to the model, to account for an inhibitor.
In general an inhibitor is a molecule that can bind to an enzyme and decrease its activity. 
An \textbf{Uncompetitive Inhibitor}  is a molecule that can only bind to an Enzyme-Substrate complex. 

\subsubsection{Mechanism}
Add a new equation to account for the Inhibitor.
Assuming it establishes an equilibrium, write down the dissociation constant for the inhibitor complex (k$_I$). 

\subsubsection{Enzyme Concentration}
Find an expression for the enzyme concentration.
State all assumptions along the way. 

\subsubsection{Product Formation Rate}
Show the rate of formation of the products simplifies to:
\be
r_p = \frac{v_{max}[S]}{k_d + [S]\left(1+\frac{[I]}{k_i}\right)}
\ee

Does your result make sense when compared to the original M-M product formation rate?

\section{Final Exam}
At this point you should start preparing for the final exam.
Expect $\approx$ $\frac{1}{3}$ of the final to be essentially the same material as previous exams and homework. 
Also expect the exam to be roughly evenly distributed over the course material (i.e. Exam 1, Exam  2, and post-Exam 2). 

A final problem set and solution covering  material in Chapter 21 will be  posted \textbf{NEXT WEEK}. 
We are waiting  to see how much material is covered in lecture before making the set. 
This final problem  set \textbf{IS} part of the course and may appear again of the final. 

\end{document}