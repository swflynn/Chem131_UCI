\documentclass{article}
\usepackage[utf8]{inputenc}

\title{Chem132A Discussion 6 Solutions}
\author{Moises Romero (moiseser@uci.edu), Shane Flynn (swflynn@uci.edu) }
\date{11/14/17}

\usepackage{graphicx}
\usepackage{amsmath}
\usepackage{braket}
\usepackage[margin=0.7in]{geometry}
\usepackage[version=4]{mhchem}
\usepackage{braket}


\newcommand{\be}{\begin{equation}}
\newcommand{\ee}{\end{equation}}
\newcommand{\pd}{\partial}

\begin{document}

\maketitle

\section{Electrochemistry}
\subsection{Problem}
Consider the cell whose potential is 1.23V :
\be
Zn(s)\big|ZnCl_2(.005\small{\frac{mol}{kg}})\big|Hg_2Cl_2(s)\big|Hg(l)
\ee
\subsection{The Reaction}
The first step is to translate the cell notation into a redox reaction. 
You'll want to do this by looking at a table of half reactions and choosing the appropriate ones. 
Once you've created your reaction determine the standard potential for the reaction and write out the Nernst equation. 

\bigskip
\textbf{Solution} 
The two half reactions are as follows : 
\be
Zn^{2+} + 2e^- \rightarrow Zn(s) \quad E^o = -.76
\ee
\be
Hg_2Cl_2(s) + 2e \rightarrow 2Hg^+ + 2Cl^- \quad E^o = .27
\ee

We know that the Zinc standard equation is on the left hand side there fore it can be flipped as : 
\be
Zn(s) \rightarrow Zn^{2+} + 2e^- 
\ee
Since we flipped this equation so we change the sign so the potential for this half reaction is .76. We then add this half reaction to the second to get : 
\be
Zn(s) + Hg_2Cl_2(s) + 2e \rightarrow Zn^{2+} + 2e^- + 2Hg^+ + 2Cl^-
\ee
We can then cancel out the electrons and our reaction is : 
\be
Zn(s) + Hg_2Cl_2(s) \rightarrow Zn^{2+}(aq) + 2Hg^+(aq) + 2Cl^-(aq)
\ee

The standard cell potential can be calculated as follows : 
\be
E^o = E^o_R - E^o_L = .27 - (-.76) = 1.03
\ee

The Nernst equation can be expressed as : 
\be
E_{cell} = 1.03 - \frac{RT}{vF}ln(Q) = 1.03 - \frac{(8.314)(298)}{v(9.648 \times 10^4)} = 1.03 - \frac{25.693mV}{v}ln(Q)
\ee


\subsection{Calculating Gibbs and K}
Calculate $\Delta_r G$, $\Delta_r G^o$ , and $K_{eq}$ for this reaction. 
This conversion factor will be useful 1CV=1J. 

\bigskip

\textbf{Solution}
\bigskip

Gibbs can be calculated as follows : 
\be
\Delta_r G = -vFE_{cell} = -(2)(9.648 \times 10^4 Cmol^{-1})(1.23V)= 238742 {CVmol^{-1}} =  - 238.7 kJmol^{-1} 
\ee

Standard Gibbs can be calculated as follows : 
\be
\Delta_r G^o = -vFE^o_{cell}= -(2)(9.648 \times 10^4)(1.03) = -238 kJmol^{-1}
\ee

K can be calculated as follows : 
\be
ln(K_{eq}) = \frac{vF}{RT}E^o_{cell} = \frac{(2)(9.648 \times 10^4)}{(8.314)(298)}(1.03) => K = e^{80.21} = 6.84 \times 10^{34}
\ee

\subsection{Activity}
Using the measured cell potential calculate the mean ionic activity coefficient and activity coefficient for $ZnCl_2$..

Calculate the mean ionic activity of $ZnCl_2$ using the Debye-Huckel limiting law.

\bigskip
\textbf{Using Measured Cell Potential Solution:}
First we want to develop an expression for Q, we know from the redox reaction calculated in 1.4, that the two active species are the Zinc ion (Z$^{+2}$) and Chlorine ion (Cl$^-$). So we can write Q as :
\be
Q=a(Zn^{+2})a^2(Cl^-)
\ee
We know that activity (a) can be expressed as : 
\be
a=\gamma_c x = \gamma_c \frac{b}{b^o}
\ee
Where subscript c represents the ion charge and x is a mole fraction.
We subsitite our defintion for activity into our Q value to get : 
\be
Q= \left(\gamma_+\frac{b}{b^o}\right)_{Zn^{+2}}\left(\gamma_- \frac{2b}{b^o}\right)^2_{Cl^-}
\ee
We multiply the Chlorine concentrations by 2 since there is twice the amount of Chlorine ion than Zinc ion in our products. We can furthr simplify this by setting our refrence state b$^o$ = 1 , and using the following definition : 
\be
\gamma_+ \gamma_- = \gamma^2_{\pm}
\ee
With this Q is defined as : 
\be
Q= 4 \gamma^3_{\pm}b^3
\ee

We can then express this within our Nernst equation, and we know v=2 since we are subbing in two electrons so it is written as follows : 
\be
E_{cell} = 1.03 - \frac{25.693mV}{2}ln(4 \gamma^3_{\pm}b^3)
\ee

Plugging in our cell potential and given molality this expression is : 
\be
1.23V = 1.03V - \frac{25.693mV}{2}ln(4 \gamma^3_{\pm}(.005)^3)
\ee
Doing out some of the math and we are left with : 
\be
1.23V = 1.03V - .0128Vln((5\times 10^-7) (\gamma^3_{\pm}))
\ee

We can then subtract and divide the Volt terms and get the following expression: 


\be
-15.625 = ln((5\times 10^-7) (\gamma^3_{\pm}))
\ee

Activity coefficient can now be expressed as : 
\be
\gamma^3_{\pm} = \frac{e^{-15.625}}{5\times 10^-7} = .3274
\ee

We then take the cube root of this value to get 
\be
\gamma_{\pm} \equiv .700
\ee

\textbf{Debye-Huckel Solution:}

Recall the Debye-Huckel equation is 
\be
log\gamma_{\pm} = - A|z_+z_-|\sqrt{I}
\ee
and Ionic Strength (I) is calculated as : 
\be
I = \frac{1}{2}\sum_i z^2_i\left(\frac{b}{b^o}\right)
\ee
Ionic Stregnth is calculated as : 
\be
I= \frac{1}{2} [(4)\times (.005)+(.01)]= .015
\ee
We then calculate the activity coefficient : 
\be
log\gamma_{\pm} = - (.504)|-2|\sqrt{.015} = -.125
\ee
\be
\gamma_{\pm} = 10^{-.125} \equiv .700
\ee


\subsection{Enthalpy and Entropy}

Given that :
\be
\left(\frac{\pd E_{cell}}{\pd T}\right)_P = -4.52 x 10^{-4} V K^{-1} 
\ee
Calculate $\Delta_r S$ and $\Delta_r H$. 

\textbf{Entropy Solution}
Using the fundamental equations for Gibbs at constant P we get : 
\be
\Delta_rS=-\left(\frac{\pd \Delta_rG}{\pd T}\right)_P
\ee
We know that Gibbs can be described as equation (3) for Cell potential , so we can rewrite the Maxwell relation and calculate Entropy(1CV=1K) : 
\be
\Delta_rS = vF\left(\frac{\pd E_{cell}}{\pd T}\right)_P= (2)(9.64 \times 10^4 Cmol^{-1})(-4.52 \times 10^{-4}VK^{-1} = 87JK^{-1}mol^{-1}
\ee

\textbf{Enthalpy Solution:}
We can now easily calculate enthalpy using : 
\be
\Delta_rH=\Delta_rG+T\Delta_rS= (-236kJmol^{-1}) + (298K)(-.087kJK^{-1}mol^{-1}) = -262kJmol^{-1}
\ee

\section{Space}
Let's assume that interstellar space contains 10 Helium atoms per cubic centimeter, and the average temperature of space is 2.7K. 

\subsection{Pressure}
Is assuming a gas in space to be ideal a physically reasonable assumption?

Derive an expression for the pressure of a Virial gas up to the second virial coefficient. 
Use this equation to compute the partial pressure of Helium in interstellar space (in Torr). 

Next do the computation using the ideal gas law.

\subsection*{Solution}
Of-course a gas in space can be approximated as an ideal gas (space is almost vacuum!). 

To prove this we will do both calculations. 

For an ideal gas we know that
\be
P = \frac{RT}{V_m} = 2.79 \times 10^{-18}\text{torr}
\ee
As expected the pressure is extremely small because ya know... space. 

If we repeat the calculation using the Virial equation we find
\be
P = \frac{RT}{V_m}\left(1+\frac{B}{V_m}\right) = 2.79 \times 10^{-18}\text{torr}
\ee
At 3 Kelvin, the Virial coefficient of helium is -0.141. 

And as expected we ge the same result as the ideal gas equation. 
\subsection{Speed}
Compute the average speed, most probable speed, and the root mean squared speed for the Helium atoms in space. 

\subsection*{Solution}
Nothing special here, we simply use the analytical expressions found in the book for each equation. 
\be
\begin{split}
v_{mp} &= \sqrt{\left(\frac{2RT}{M}\right)} = 105(m/s)\\
\braket{v} &=\sqrt{\left(\frac{8RT}{\pi M}\right)} = 119(m/s)\\
v_{rms} &= \sqrt{\left(\frac{3RT}{M}\right)} = 129(m/s)
\end{split}
\ee

\subsection{Collisions}
How far will a single Helium atom travel before colliding with another Helium?

\subsection*{Solution}
\be
\lambda = \frac{c_{rel}}{z} = \frac{kT}{\sigma P} = \frac{kT}{\pi d^2 P} = 4.78 \times 10^{11}m
\ee












\end{document}