\documentclass{article}
\usepackage[utf8]{inputenc}

\title{Chem132A Discussion 6 Homework}
\author{Moises Romero (moiseser@uci.edu), Shane Flynn (swflynn@uci.edu) }
\date{11/4/17}

\usepackage{graphicx}
\usepackage{amsmath}
\usepackage{braket}
\usepackage[margin=0.7in]{geometry}
\usepackage[version=4]{mhchem}


\newcommand{\be}{\begin{equation}}
\newcommand{\ee}{\end{equation}}
\newcommand{\pd}{\partial}

\begin{document}

\maketitle

\section{Electrochemistry}
Reactions where electrons are transferred between molecules or atoms are called Redox (oxidation-reduction) reactions. 
Redox reactions can be expressed as the difference between two half reactions, highlighting the electron transfers occurring in the system. 

\subsection*{Standard Potentials}
Standard potentials are typically measured relative to the standard hydrogen electrode (SHE), which is assigned a potential of 0. 
\be
Pt(s)|H_2(g)|H^+(aq) E^o=0
\ee
Electrode notation is read from left to right, $L||R$. Therefore the standard potential for a redox reaction can be calculated as follows: 
\be
L||R => E_{cell}^o = E^o (R) - E^o (L)
\ee
\subsection*{Nernst Equation}
We can relate the Gibbs Free Energy to the cell potential:
\be
\Delta_r G = -vFE_{cell}
\ee
Where $v$ is the stoichiometric coefficient of the half reactions, F is Faraday's Constant = 9.648 x 10$^4$ $\frac{C}{mol}$ . 

Because the Gibbs Free Energy is a thermodynamic potential, we can relate other expressions for G to the Nernst Equation. 
\be
\Delta_r G =\Delta_r G^o + RTln(Q) => -vFE_{cell} = -vFE^o_{cell} + RTln(Q) 
\ee
We can divide this equation by -vF to get : 
\be
E_{cell} = E^o_{cell} - \frac{RT}{vF}ln(Q)
\ee
We know G=0 at equilibrium, therefore the potential must also be 0.
At equilibrium we can therefore write: 
\be
\ln(K_{eq}) = \frac{nF}{RT}E^o_{cell}
\ee

This equation indicates that :

\bigskip

When $K>1 , E^{o}>0$ reaction favors production formation

\bigskip
When $K<1 , E^{o}<0$ reaction favors production formation

\subsection*{Electrochemical Analysis}
Consider the cell whose potential is 1.23V :
\be
Zn(s)\big|ZnCl_2(.005\small{\frac{mol}{kg}})\big|Hg_2Cl_2(s)\big|Hg(l)
\ee
\subsection{The Reaction}
The first step is to translate the cell notation into a Redox Reaction. 
You'll want to do this by looking at a table of half reactions. 
Once you've created your total reaction from the half reactions, determine the standard potential for the reaction and write out the Nernst equation 

\subsection{Calculating Gibbs and K}
Calculate $\Delta_r G$, $\Delta_r G^o$ , and $K_{eq}$ for this reaction. 

\subsection{Activity}
Using the measured cell potential calculate the mean ionic activity coefficient and activity coefficient for ZnCl$_2$.
Calculate the mean ionic activity of ZnCl$_2$ using the Debye-Huckel limiting law 

\subsection{Enthalpy and Disorder}
Given that :
\be
\left(\frac{\pd E_{cell}}{\pd T}\right)_P = -4.52 x 10^{-4} V K^{-1} 
\ee
Calculate $\Delta_r S$ and $\Delta_r H$. 

\section{Space}
Let's assume that interstellar space contains 10 Helium atoms per cubic centimeter, and the average temperature of space is 2.7K. 

\subsection{Pressure}
Is assuming a gas in space to be ideal a physically reasonable assumption?

Derive an expression for the pressure of a Virial gas up to the second virial coefficient. 
Use this equation to compute the partial pressure of Helium in interstellar space (in Torr). 

Next do the computation using the ideal gas law.

\subsection{Speed}
Compute the average speed, most probable speed, and the root mean squared speed for the Helium atoms in space. 

\subsection{Collisions}
How far will a single Helium atom travel before colliding with another Helium?

\section{Exam Review}
As with Exam 1, compose a study guide for Exam 2. 
Note: this course is cumulative by nature, you need to go back and review your study guide for exam 1!
If you did not make a study guide for exam 1.... go do that too. 

What are the major topics that will potentially be covered on Exam 2?

\end{document}